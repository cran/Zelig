\section{{\tt rocplot}: Receiver Operator Characteristic Plots}\label{ss:rocplot}

\subsubsection{Description}
The \texttt{rocplot()} command generates a receiver operator
characteristic plot to compare the in-sample (default) or
out-of-sample fit for two logit or probit regressions.  

\subsubsection{Syntax}
\begin{verbatim}
rocplot(y1, y2, fitted1, fitted2, cutoff = seq(from=0, to=1, length=100), 
        lty1 = "solid", lty2 = "dashed", lwd1 = par("lwd"), lwd2 = par("lwd"),
        col1 = par("col"), col2 = par("col"), plot = TRUE, ...)
\end{verbatim}

\subsubsection{Arguments}
\begin{itemize}
\item {\tt y1}: Response variable for the first model.  
\item {\tt y2}: Response variable for the second model. 
\item {\tt fitted1}: Fitted values for the first model.  These values
  may represent either the in-sample or out-of-sample fitted values.
\item {\tt fitted2}: Fitted values for the second model.  
\item {\tt cutoff}: A vector of cut-off values between 0 and 1, at
  which to evaluate the proportion of 0s and 1s correctly predicted by
  the first and second model.  By default, this is 100 increments
  between 0 and 1, inclusive.  
\item {\tt lty1}, {\tt lty2}: The line type for the first model ({\tt
    lty1}) and the second model ({\tt lty2}), defaulting to solid and
  dashed, respectively.    
\item {\tt lwd1}, {\tt lwd2}: The width of the line for the first
  model ({\tt lwd1}) and the second model ({\tt lwd2}), defaulting to
  1 for both.  
\item {\tt col1}, {\tt col2}: The colors of the line for the first
  model ({\tt col1}) and the second model ({\tt col2}), defaulting to
  black for both.
\item{\tt plot}: By default, {\tt plot = TRUE}, which generates a plot to the
  selected device.  If {\tt FALSE}, {\tt rocplot()} returns a list of output (see below).  
\item {\tt \dots}: Additional parameters passed to plot, including
  {\tt xlab}, {\tt ylab}, and {\tt main}.  
\end{itemize}

\subsubsection{Output Values}

If {\tt plot = TRUE}, {\tt rocplot()} generates an ROC plot for two
logit or probit models.  You may save this plot using the commands
described in \Sref{ss:output}.

If {\tt plot = FALSE}, {\tt rocplot()} returns a list with the following
elements:
\begin{itemize}
\item {\tt roc1}:  A matrix containing a vector of x-coordinates and
  y-coordinates corresponding to the number of ones and zeros correctly
  predicted for the first model.
\item {\tt roc2}:  A matrix containing a vector of x-coordinates and
  y-coordinates corresponding to the number of ones and zeros correctly
  predicted for the second model.
\item {\tt area1}:  The area under the first ROC curve, calculated using
  Reimann sums.
\item {\tt area2}:  The area under the second ROC curve, calculated using
  Reimann sums.
\end{itemize}

\subsubsection{Example}
You may view this example using {\tt demo(roc)}.  
\begin{verbatim}
> data(turnout)
> z.out1 <- zelig(vote ~ race + educate + age, model = "logit", 
                  data = turnout)
> z.out2 <- zelig(vote ~ race + educate, model = "logit", 
                  data = turnout)
> rocplot(z.out1$y, z.out2$y, fitted(z.out1), fitted(z.out2))
\end{verbatim}

\subsubsection{See Also}
\begin{itemize}
  \item The {\tt help(plot)} and {\tt help(lines)} reference pages.  
  \item \Sref{s:plot} for an overview of plotting procedures.
\end{itemize}

\subsubsection{Contributors}

Kosuke Imai, Gary King, and Olivia Lau.  

Sample data are a selection of $2,000$ observations from
\begin{verse}
\bibentry{KinTomWit00}.  
\end{verse}
%%% Local Variables: 
%%% mode: latex
%%% TeX-master: t
%%% End: 

