\section{{\tt plot.ci}: Plotting Vertical confidence Intervals}\label{plot.ci}

\subsubsection{Description}
The \texttt{plot.ci()} command generates vertical confidence intervals
for linear or generalized linear univariate response models.  

\subsubsection{Syntax}
\begin{verbatim}
> plot.ci(object, CI = 95, qi = "ev", col = c("red", "blue"), ...) 
\end{verbatim}

\subsubsection{Arguments}
\begin{itemize}
\item {\tt object}: stored output from {\tt sim()}.  The {\tt x} and optional
  {\tt x1} values used to generate the {\tt sim()} output object must
  have more than one observation.  
\item {\tt CI}: the selected confidence interval.  Defaults to 95
  percent.  
\item {\tt qi}: the selected quantity of interest.  Defaults to
  expected values.
\item {\tt col}: a vector of at most two colors for plotting the
  expected value given by {\tt x} and the alternative set of expected
  values given by {\tt x1} in {\tt sim()}.  If the quantity of
  interest selected is not the expected value, or {\tt x1 = NULL},
  only the first color will be used. 
\item {\tt \dots}: Additional parameters, such as {\tt xlab}, {\tt
    ylab}, and {\tt main}, passed to {\tt plot()}.  
\end{itemize}

\subsubsection{Output Values}
For all univariate response models, {\tt plot.ci()} returns vertical
confidence intervals over a specified range of one explanatory
variable.  You may save this plot using the commands described in
\Sref{ss:output}.  

\subsubsection{Example}

This demo is available within R as {\tt demo(vertci)}.  Load the
sample data and estimate the model.
\begin{verbatim}
data(turnout)
z.out <- zelig(vote ~ race + educate + age + I(age^2) + income,
               model = "logit", data = turnout)
\end{verbatim}
Establish a range for {\tt `age'}, the key explanatory variable, and
create two {\tt setx} objects that hold the other explanatory
variables to fixed, while allowing age to range from 18 to 95, inclusive.
\begin{verbatim}
age.range <- 18:95
x.low <- setx(z.out, educate = 12, age = age.range)
x.high <- setx(z.out, educate = 16, age = age.range)
\end{verbatim}
Simulate quantities of interest.  
\begin{verbatim}
s.out <- sim(z.out, x = x.low, x1 = x.high)
\end{verbatim}
Plot these results using the {\tt plot.ci()} function:  
\begin{verbatim}
plot.ci(s.out, xlab = "Age in Years",
        ylab = "Predicted Probability of Voting",
        main = "Effect of Education and Age on Voting Behavior")
legend(45, 0.52, legend = c("College Education (16 years)",
       "High School Education (12 years)"), col = c("blue","red"), 
       lty = c("solid"))
\end{verbatim}

\subsubsection{See Also}
\begin{itemize}
  \item The {\tt help(plot)} and {\tt help(lines)} reference pages.    
  \item \Sref{s:plot} for an overview of plotting procedures
\end{itemize}

\subsubsection{Contributors}

Kosuke Imai, Gary King, and Olivia Lau created the R procedure to
generate vertical confidence intervals for simulated quantities of
interest.  

Sample data are a selection of $2,000$ observations from
\begin{verse}
\bibentry{KinTomWit00}.  
\end{verse}
%%% Local Variables: 
%%% mode: latex
%%% TeX-master: t
%%% End: 
