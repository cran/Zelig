\section{{\tt set.start}: Set starting values for all parameters}
\label{set.start}

\subsubsection{Description}

After using {\tt parse.par()} and {\tt model.matrix()}, use {\tt
set.start()} to set starting values for all parameters.  By default,
starting values are set to 0.  If you wish to select alternative
starting values for certain parameters, use {\tt put.start()} after
{\tt set.start()}.

\subsubsection{Syntax}
\begin{verbatim}
set.start(start.val = NULL, terms)
\end{verbatim}

\subsubsection{Arguments}

\begin{itemize}
\item {\tt start.val}: user-specified starting values.  If {\tt NULL}
(default), the default starting values for all parameters are set to
0.  
\item {\tt terms}: the terms output from {\tt
model.frame.multiple()}.  
\end{itemize}

\subsubsection{Output Values}
A named vector of starting values for all parameters specified in {\tt
terms}, defaulting to 0.  

\subsubsection{Example}
\begin{verbatim}
fml <- parse.formula(formula, model = "bivariate.probit")
D <- model.frame(fml, data = data)
terms <- terms(D)
start.val <- set.start(start.val = NULL, terms)
\end{verbatim}

\subsubsection{See Also}
\begin{itemize}
\item \Sref{put.start} to change starting values for specific
parameter sets.  
\item \Sref{s:new} for detailed examples of writing new models.  
\end{itemize}

\subsubsection{Contributors}

Kosuke Imai, Gary King, Olivia Lau and Ferdinand Alimadhi.

%%% Local Variables: 
%%% mode: latex
%%% TeX-master: "~/zelig/docs/zelig"
%%% End: 
