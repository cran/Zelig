\section{{\tt mi}: Create a list of multiply imputed data frames}
\label{mi.command}

\subsubsection{Description}

Use {\tt mi()} to create a list of multiply imputed data frames for
use with {\tt zelig()}.  

\subsubsection{Syntax}
\begin{verbatim}
mi(data1, data2, ...)
\end{verbatim}

\subsubsection{Arguments}

\begin{itemize}
\item {\tt data1}:  the first multiply imputed data frame.
\item {\tt data2}: the second multiply imputed data frame.
\item {\tt \dots}: additional multiply imputed data frames.
\end{itemize}

\subsubsection{Output Values}
A list of data frames with class {\tt "mi"}
\subsubsection{See Also}
\begin{itemize}
\item See \Sref{mi.ex} to use {\tt mi()} with {\tt zelig()}.  
\end{itemize}

\subsubsection{Contributors}

Kosuke Imai, Gary King, Olivia Lau, and Ferdinand Alimadhi.

%%% Local Variables: 
%%% mode: latex
%%% TeX-master: "~/zelig/docs/zelig"
%%% End: 
