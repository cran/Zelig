\section{{\tt ternarypoints}: Adding Points to Ternary Diagrams}
\label{ternarypoints}

\subsubsection{Description}
Use {\tt ternarypoints()} to add points to a ternary diagram generated
using the {\tt ternaryplot()} function in the vcd library.  Use
ternary diagrams to plot expected values for multinomial choice models
with three categories in the dependent variable.  

\subsubsection{Syntax}
\begin{verbatim}
> ternarypoints(object, pch = 19, col = "blue", ...)
\end{verbatim}

\subsubsection{Arguments}
\begin{itemize}
\item {\tt object}: The input object must be a matrix with three
  columns.   
\item {\tt pch}: The selected type of point.  By default, {\tt pch =
    19}, solid disks.  
\item {\tt col}: The color of the points.  By default, {\tt col =
    "blue"}.  
\item {\tt \dots}: Additional parameters passed to {\tt points()}.  
\end{itemize}

\subsubsection{Output Values}
The {\tt ternarypoints()} command adds points to a previously existing
ternary diagram.  Use {\tt ternaryplot()} to generate the main ternary diagram.  

\subsubsection{Examples}
\begin{verbatim}
> ternaryplot(s.out$qi$ev)
> ternarypoints((s.out$qi$ev + s.out$qi$fd))
\end{verbatim}

\subsubsection{See Also}
\begin{itemize}
  \item \Sref{ternaryplot} to create a ternary diagram, to which you
may add points.  
  \item \Sref{s:plot} for an overview of plotting procedures
\end{itemize}

\subsubsection{Contributors}

Kosuke Imai, Gary King, and Olivia Lau created the {\tt
  ternarypoints()} function to work with {\tt ternaryplot()}.    

%%% Local Variables: 
%%% mode: latex
%%% TeX-master: t
%%% TeX-master: t
%%% End: 











