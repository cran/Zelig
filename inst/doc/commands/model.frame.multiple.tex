\section{{\tt model.frame.multiple}: Extracting the ``environment'' of
a model formula}
\label{model.frame.multiple}

\subsubsection{Description}
Use {\tt model.frame.multiple()} after {\tt parse.par()} to create a
data frame of the unique variables identified in the formula (or list
of formulas).  
  
\subsubsection{Syntax}
\begin{verbatim}
model.frame.multiple(formula, data, eqn = NULL, ...)
\end{verbatim}

\subsubsection{Arguments}
\begin{itemize}
  \item {\tt formula}: a list of formulas of class {\tt "multiple"},
returned from {\tt parse.par()}.  
  \item {\tt data}: a data frame containing all the variables used in
{\tt formula}.  
\item {\tt eqn}: an optional character string or vector of character strings 
specifying the equations for which you would like to extract variables.  Defaults 
to {\tt NULL}, which pulls out all the variables for all equations in {\tt 
formula}.   
 \item {\tt \dots}: additional arguments passed to {\tt
model.frame.default()}.  
\end{itemize}

\subsubsection{Output Values}

The output is a data frame (with a terms attribute) containing all the
unique explanatory and response variables identified in the list of
formulas.  By default, missing ({\tt NA}) values are listwise deleted.

If {\tt as.factor()} appears on the left-hand side, the response
variables will be returned as an indicator (0/1) matrix with columns
corresponding to the unique levels in the factor variable.  
	
If any formula contains a {\tt tag()}, {\tt model.frame.multiple()}
will return the original variable in the data frame and use the {\tt
tag()} information in the terms attribute only.

\subsubsection{Examples}
\begin{verbatim}
formulae <- list(import ~ coop + cost + target,
                 export ~ coop + cost + target)
fml <- parse.formula(formulae, model = "bivariate.logit")
D <- model.frame(fml, data = mydata)
\end{verbatim}
Since the output from {\tt parse.formula()} is of class {\tt
"multiple"}, you do not need to call {\tt model.frame.multiple()}
explicitly, but can use the generic {\tt model.frame()} instead.    

\subsubsection{See Also}
\begin{itemize}
\item \Sref{parse.formula} for {\tt parse.formula()}
\item \Sref{ui} for an overview of the user-interface.  
\end{itemize}

\subsubsection{Contributors}

Kosuke Imai, Gary King, Olivia Lau, and Ferdinand Alimadhi.


%%% Local Variables: 
%%% mode: latex
%%% TeX-master: t
%%% End: 












