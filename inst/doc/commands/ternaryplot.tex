\section{{\tt ternaryplot}: Ternary Diagram for 3D Data}
\label{ternaryplot}

\subsubsection{Description} 
Visualizes compositional, 3-dimensional data in an equilateral
triangle.   A point's coordinates are found by computing the gravity center
of mass points using the data entries as weights. Thus, the coordinates
of a point $P(a,b,c)$ for $a + b + c = 1$ are: $P(b + \frac{c}{2}, \frac{c
\sqrt{3}}{2})$.

\subsubsection{Syntax}
\begin{verbatim}
ternaryplot(x, scale = 1, dimnames = NULL, 
            dimnames.position = c("corner","edge","none"),
            dimnames.color = "black", id = NULL, id.color = "black", 
            coordinates = FALSE, grid = TRUE, grid.color = "gray", 
            labels = c("inside", "outside", "none"), 
            labels.color = "darkgray", border = "black", bg = "white",
            pch = 19, cex = 1, prop.size = FALSE, col = "red", 
            main = "ternary plot", ...)
\end{verbatim}

\subsubsection{Arguments}
\begin{itemize}
\item {\tt x}: a matrix with three columns.
\item {\tt  scale}: row sums scale to be used.
\item {\tt dimnames}: dimension labels (defaults to the column names
of {\tt x}).
\item {\tt dimnames.position, dimnames.color}: position and color of dimension labels.
\item {\tt id}: optional labels to be plotted below the plot
symbols. {\tt coordinates} and {\tt id} are mutual exclusive. 
\item {\tt id.color}:  color of these labels.
\item {\tt coordinates}: if {\tt TRUE}, the coordinates of the points are
    plotted below them. {\tt coordinates} and {\tt id} are mutual exclusive.
\item {\tt grid}: if {\tt TRUE}, a grid is plotted. May optionally
    be a string indicating the line type (default: {\tt "dotted"}).
\item {\tt grid.color}: grid color.
\item {\tt labels, labels.color}: position and color of the grid labels.
\item {\tt border}: color of the triangle border.
\item {\tt bg}: triangle background.
\item {\tt pch}: plotting character. Defaults to filled dots.
\item {\tt cex}: a numerical value giving the amount by which plotting text
    and symbols should be scaled relative to the default. Ignored for
    the symbol size if {\tt prop.size} is not {\tt FALSE}.
\item {\tt prop.size}: if {\tt TRUE}, the symbol size is plotted
    proportional to the row sum of the three variables, i.e. represents
    the weight of the observation.
\item {\tt col}: plotting color.
  \item {\tt main}: main title.
\item {\tt \dots}: additional graphics parameters (see {\tt par}).
\end{itemize}

\subsubsection{Examples}
\begin{verbatim}
data(mexico)
z.out <- zelig(as.factor(vote88) ~ pristr + othcok + othsocok, 
                model = "mlogit", data = mexico)
x.out <- setx(z.out)
s.out <- sim(z.out, x = x.out)

ternaryplot(s.out$qi$ev, pch = ".", col = "blue",
            main = "1988 Mexican Presidential Election")
\end{verbatim}

\subsubsection{See Also}
\begin{itemize}
\item \Sref{ternarypoints} for a way to add points to an existing
ternary diagram.  
\item \Sref{s:plot} for an overview of plotting procedures.  
\end{itemize} 

\subsubsection{Contributors}

This function was originally written by David Meyer for the vcd
library, version 0.1, which uses plot graphics, and differs from
the current implementation in vcd, which uses grid graphics.
For additional information on vcd, please consult
\begin{verse}
\bibentry{Friendly00}.
\end{verse}

