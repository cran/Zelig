\section{{\tt setx}: Setting Explanatory Variable Values}\label{ss:setx}

\subsubsection{Description}
The \texttt{setx()} command uses the variables identified in the
\texttt{formula} generated by \texttt{zelig()} and sets the values of
the explanatory variables to the selected values.  Use \texttt{setx()}
after {\tt zelig()} and before \texttt{sim()} to simulate quantities
of interest.

\subsubsection{Syntax}
\begin{verbatim}
> x.out <- setx(z.out, fn = list(numeric = mean, ordered = median, 
                other = mode), data = NULL, cond = FALSE, ...)
\end{verbatim}

\subsubsection{Arguments}
\begin{itemize}
  
\item {\tt z.out}: the saved output from \texttt{zelig()}.
  
\item {\tt fn}: a list of three functions to apply to three types of
  variables
  \begin{itemize}
  \item {\tt numeric} variables are set to their mean by default, but
    you may select any other mathematical function.
  \item {\tt ordered} factors are set to their median by default, and
    most mathematical operations will work on them.  (If you select
    {\tt ordered = mean}, however, {\tt setx()} will default to median
    with a warning.)
  \item {\tt other} variables may consist of unordered factors,
    character strings, or logical variables.  The {\tt other}
    variables may only be set to their mode.  (If you wish to set one
    of the other variables to a specific value, you may do so using
    {\tt \dots} or by making them numeric and using \texttt{fn}.)
  \end{itemize}
  In the special case {\tt fn = NULL}, {\tt setx()} will return all of
  the observations without applying any function to the data.
\item {\tt data}: a new data frame used to set the values of
  explanatory variables. If {\tt data = NULL} (the default), the data
  frame called in \texttt{zelig()} is used.
\item {\tt cond}: a logical value indicating whether the usual
  unconditional prediction is used (the default) or whether the optional
  conditional prediction should be performed (choose {\tt cond =
    TRUE}).  If you choose {\tt cond = TRUE}, {\tt setx()} will coerce
  {\tt fn = NULL} and ignore the arguments in {\tt \dots}.
\item {\tt \dots}: user-defined values of specific variables that
  overwrite the default values set by the function.  For example,
  adding {\tt var1 = mean(data\$var1)} sets the value of \texttt{var1}
  to the mean of {\tt data\$var1}; {\tt x1 = 12} explicitly sets the value
  of {\tt x1} to 12.  In addition, you may specify one explanatory
  variable as a range of values, creating one observation for every
  unique value in the range of values.  The other variables are set to
  the values given by {\tt fn}.
\end{itemize}

\subsubsection{Output Values}
\begin{itemize}
\item For unconditional prediction, {\tt x.out} is a model matrix
  based on the specified values for the explanatory variables.  For
  multiple analyses (i.e., when choosing the {\tt by} option in {\tt
    zelig()}, {\tt setx()} returns the selected values calculated over
  the entire data frame.  If you wish to calculate values over just
  one subset of the data frame, the 5th subset for example, you may
  use:
\begin{verbatim}
> x.out <- setx(z.out[[5]])
\end{verbatim}
  
\item For conditional prediction, {\tt x.out} includes the model matrix
  and the dependent variable(s).  For multiple analyses (when choosing
  the {\tt by} option in {\tt zelig()}), {\tt setx()} returns the
  observed explanatory variables in each subset.
\end{itemize}

\subsubsection{Example: Unconditional Prediction}
\begin{verbatim}
> data(turnout) 
> z.out <- zelig(vote ~ race + educate, model = "logit", data = turnout)
> x.out <- setx(z.out)
> s.out <- sim(z.out, x = x.out)
\end{verbatim}
Using all observations:
\begin{verbatim}
> x.out <- setx(z.out, fn = NULL)
\end{verbatim}
Defining a user defined function for {\tt fn}:
\begin{verbatim}
> quants <- function(x) {
+  quantile(x, 0.25)
+ }
> x.out <- setx(z.out, fn = list(numeric = quants))
\end{verbatim} 

\subsubsection{Example: Conditional Prediction With MatchIt Data}
Use {\tt demo(match)} to view this example.  
\begin{verbatim}
> data(lalonde)
> match.out <- matchit(treat ~ age + educ + black + hispan + married + 
                       nodegree + re74 + re75, data = lalonde)
> z.out <- zelig(re78 ~ pscore, data = match.data(match, "control"), 
                 model = "ls")
> x.out <- setx(z.out, data = match.data(match, "treat"), fn = NULL, 
                cond = TRUE)
> s.out <- sim(z.out, x = x.out)
\end{verbatim}

\subsubsection{Example: Conditional Prediction With Multiple Analyses}
Use {\tt demo(conditional)} to view this example:  
\begin{verbatim}
> data(turnout)
> z.out <- zelig(vote ~ age + educate + income, by = "race",
                 data = turnout, model = "probit")

# Calculate the ATE (average treatment effect) for race == "white":
> x.white <- setx(z.out$others, data = turnout[turnout$race == "white",], 
                  fn = NULL, cond = TRUE)
> s.others <- sim(z.out$others, x = x.white)

# Calculate the ATE for race == "others":
> x.others <- setx(z.out$white, data = turnout[turnout$race == "others",], 
                  fn = NULL, cond = TRUE)
> s.others <- sim(z.out$white, x = x.others)
\end{verbatim} %$

\subsubsection{See Also}
  \begin{itemize}
  \item \Sref{s:commands} for an overview of the Zelig simulation
    procedure.
  \item \Sref{s:models} for an overview of supported models.
  \item \Sref{s:syntax} for an overview of R syntax.
\end{itemize}

\subsubsection{Contributors}

Kosuke Imai, Gary King, and Olivia Lau ported {\tt setx()} to R, following
the logic of Clarify (for Stata).  See King, Tomz, and Wittenberg,
2000.\nocite{KinTomWit00}

Sample data are a selection of $2,000$ observations from

\begin{verse}
\bibentry{KinTomWit00}.
\end{verse}


%%% Local Variables: 
%%% mode: latex
%%% TeX-master: t
%%% End: 
