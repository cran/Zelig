\section{{\tt sim}: Simulating Quantities of Interest}\label{ss:sim}

\subsubsection{Description}
Simulate quantities of interest from the estimated model output from
\texttt{zelig()} given specified values of explanatory variables
established in \texttt{setx()}.  For classical \emph{maximum
  likelihood} models, {\tt sim()} uses asymptotic normal approximation
to the log-likelihood.  For \emph{Bayesian models}, Zelig simulates
quantities of interest from the posterior density, whenever possible.
For \emph{robust Bayesian models}, simulations are drawn from the
identified class of Bayesian posteriors.  Alternatively, you may
generate quantities of interest using bootstrapped parameters.  

\subsubsection{Syntax}
\begin{verbatim}
> s.out <- sim(z.out, x, x1 = NULL, num = c(1000, 100), prev = NULL,
               bootstrap = FALSE,  bootfn = NULL, cond.data = NULL, ...)
\end{verbatim}

\subsubsection{Arguments}
\begin{itemize}
  \item {\tt z.out}: the output object from \texttt{zelig()}.
  \item {\tt x}: values of explanatory variables used
    for simulation, generated by {\tt setx()}.  
  \item {\tt x1}: optional values of explanatory variables (generated
    by a second call of {\tt setx()}), used to simulate first
    differences and risk ratios.  (Not available for conditional prediction.)
  \item {\tt num}: the number of simulations, i.e., posterior draws.
    If the \texttt{num} argument is omitted, \texttt{sim()} draws 1,000
    simulations if {\tt bootstrap = FALSE} (the default), or
    100 simulations if {\tt bootstrap = TRUE}.  You may increase this
    value to improve accuracy.  (Not available for conditional prediction.)
  \item {\tt bootstrap:} a logical value indicating if parameters
    should be generated by re-fitting the model for bootstrapped
    data, rather than from the likelihood or posterior.  (Not
    available for conditional prediction.)
  \item {\tt bootfn:} a function which governs how the data is
    sampled, re-fits the model, and returns the bootstrapped model
    parameters.  If {\tt bootstrap = TRUE} and {\tt bootfn = NULL},
    {\tt sim()} will sample observations from the original data (with
    replacement) until it creates a sampled dataset with the same
    number of observations as the original data.  Alternative
    bootstrap methods (for which users have to specify their own {\tt
bootfn}) include sampling the residuals rather than the
    observations, weighted sampling, and parametric bootstrapping.
    (Not available for conditional prediction.)  
  \item {\tt \dots}: additional optional arguments passed to
    \texttt{boot()}.  
\end{itemize}

\subsubsection{Output Values}
The output stored in {\tt s.out} varies by model.  Use the {\tt
  names()} command to view the output stored in {\tt s.out}.  Common
elements include:
\begin{itemize}
  \item {\tt x}: the {\tt setx()} values for the explanatory variables,
    used to calculate the quantities of interest (expected values,
    predicted values, etc.).    
  \item {\tt x1}: the optional {\tt setx()} object used to simulate
    first differences, and other model-specific quantities of
    interest, such as risk-ratios.  
  \item {\tt call}: the options selected for {\tt sim()}, used to
    replicate quantities of interest.  
  \item {\tt zelig.call}: the original command and options for {\tt
      zelig()}, used to replicate analyses.
  \item {\tt num}: the number of simulations requested.  
  \item {\tt par}: the parameters (coefficients, and additional
    model-specific parameters).  You may wish to use the same set of
    simulated parameters to calculate quantities of interest rather
    than simulating another set.
  \item {\tt qi\$ev}: simulations of the expected values given the
    model and {\tt x}.
  \item {\tt qi\$pr}: simulations of the predicted values given by the
    fitted values.
  \item {\tt qi\$fd}: simulations of the first differences (or risk
    difference for binary models) for the given {\tt x} and {\tt x1}.
    The difference is calculated by subtracting the expected values
    given {\tt x} from the expected values given {\tt x1}.  (If do not
    specify {\tt x1}, you will not get first differences or risk
    ratios.)
  \item {\tt qi\$rr}: simulations of the risk ratios for binary and
    multinomial models.  See specific models for details.    
  \item {\tt qi\$ate.ev}: simulations of the average expected
    treatment effect for the treatment group, using conditional
    prediction. Let $t_i$ be a binary explanatory variable defining
    the treatment ($t_i=1$) and control ($t_i=0$) groups.  Then the
    average expected treatment effect for the treatment group is
    \begin{equation*} \frac{1}{\sum_{i=1}^n t_i}\sum_{i:t_i=1}^n \{  Y_i(t_i=1) -
      E[Y_i(t_i=0)] \},
    \end{equation*} 
    where $Y_i(t_i=1)$ is the value of the dependent variable for
    observation $i$ in the treatment group.  Variation in the
    simulations are due to uncertainty in simulating $E[Y_i(t_i=0)]$,
    the counterfactual expected value of $Y_i$ for observations in the
    treatment group, under the assumption that everything stays the
    same except that the treatment indicator is switched to $t_i=0$.
  \item {\tt qi\$ate.pr}: simulations of the average predicted
    treatment effect for the treatment group, using conditional
    prediction. Let $t_i$ be a binary explanatory variable defining
    the treatment ($t_i=1$) and control ($t_i=0$) groups.  Then the
    average predicted treatment effect for the treatment group is
    \begin{equation*} \frac{1}{\sum_{i=1}^n t_i}\sum_{i:t_i=1}^n \{ Y_i(t_i=1) -
      \widehat{Y_i(t_i=0)} \},
    \end{equation*} 
    where $Y_i(t_i=1)$ is the value of the dependent variable for
    observation $i$ in the treatment group.  Variation in the
    simulations are due to uncertainty in simulating
    $\widehat{Y_i(t_i=0)}$, the counterfactual predicted value of
    $Y_i$ for observations in the treatment group, under the
    assumption that everything stays the same except that the
    treatment indicator is switched to $t_i=0$.
\end{itemize}

In the case of censored $Y$ in the exponential, Weibull, and lognormal
models, {\tt sim()} first imputes the uncensored values for $Y$ before
calculating the ATE.  

You may use the {\tt \$} operator to extract any of the
above from {\tt s.out}.  For example, {\tt s.out\$qi\$ev} extracts the
simulated expected values.

\subsubsection{See Also}
\begin{itemize}
  \item \Sref{s:commands} for an overview of the Zelig simulation procedure.
  \item \Sref{s:models} for an overview of supported models.  
  \item \Sref{s:syntax} for an overview of R syntax.  
\end{itemize}

\subsubsection{Contributors}

Kosuke Imai, Gary King, and Olivia Lau generalized {\tt sim()} from a
similar procedure used in Clarify (for Stata, see King, Tomz, and
Wittenberg, 2000) and added procedures to bootstrap quantities of
interest, and perform conditional prediction.  


%%% Local Variables: 
%%% mode: latex
%%% TeX-master: t
%%% End: 












