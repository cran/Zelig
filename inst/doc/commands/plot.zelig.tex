\section{{\tt plot}: Graphing Quantities of Interest}\label{ss:plot.zelig}

\subsubsection{Description}
The \texttt{plot()} command generates default plots for {\tt
  sim()} objects with one-observation values in {\tt x} and {\tt x1}.
  Additional default plotting methods for other data structures are
  described in {\tt help(plot)}.  

\subsubsection{Syntax}
\begin{verbatim}
> plot(s.out, alt.col = "red", user.par = FALSE, ...)
\end{verbatim}

\subsubsection{Arguments}
\begin{itemize}
\item{\tt s.out}: stored output from {\tt sim()}.  If the {\tt x} or
  {\tt x1} values stored in the object contain more than one
  observation, {\tt plot.zelig()} will return an error.  For linear or
  generalized linear models with more than one observation in {\tt x}
  and optionally {\tt x1}, you may use {\tt plot.ci()} (see \Sref{plot.ci}).
\item{\tt alt.col}: alternative color used for contrast in the plots.
  The primary color is black.  Type {\tt colors()} to see a list of
  available alternative colors.
\item{\tt user.par}: a logical value indicating whether to use the
  default Zelig plotting parameters ({\tt user.par = FALSE}) or
  user-defined parameters ({\tt user.par = TRUE}), set using the {\tt
    par()} function prior to plotting.
\item{\tt \dots}: Additional parameters passed to {\tt plot()}.
  Because {\tt plot.zelig()} primarily produces diagnostic plots, many
  of these parameters are hard-coded for convenience and
  presentation.  If you wish to produce plots using different
  parameters, you may follow the directions in \Sref{s:plot}.  
\end{itemize}

\subsubsection{Output Values}
Depending on the class of model selected, \texttt{plot.zelig()} will
return an on-screen window with graphs of the various quantities of
interest.  You may save these plots using the commands described in
\Sref{ss:output}.  

\subsubsection{Examples}
\begin{verbatim}
> z.out <- zelig(y ~ x1 + x2, model = "logit", data = mydata)
> x.out <- setx(z.out, x1 = 10)
> x.alt <- setx(z.out, x1 = 0)
> s.out <- sim(z.out, x = x.out, x1 = x.alt)
> plot(s.out)
\end{verbatim}

\subsubsection{See Also}
\begin{itemize}
\item The {\tt help(plot)} and {\tt help(lines)} reference pages for
  optional arguments passed to {\tt \dots}.
\item The {\tt help(par)} reference page for user-specified plotting
  parameters.     
\item \Sref{s:plot} for an overview of plotting procedures
\end{itemize}

\subsubsection{Contributors}

Kosuke Imai, Gary King, and Olivia Lau created the plots generated by
{\tt plot.zelig()}.  These plots are individually available in the
base, MASS, and vcd libraries.  

%%% Local Variables: 
%%% mode: latex
%%% TeX-master: t
%%% End: 
