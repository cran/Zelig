\section{{\tt model.end}: Cleaning up after optimization}
\label{model.end}

\subsubsection{Description}

The {\tt model.end()} function creates a list of regression output from
{\tt optim()} output.  The list includes coefficients (from the {\tt optim() 
par} output), a variance-covariance matrix (from the {\tt optim()} Hessian output), 
and any terms, contrasts, or xlevels (from the model frame).  Use  {\tt 
model.end()} after calling {\tt optim()}, but before assigning a class to the 
regression output.  

\subsubsection{Syntax}
\begin{verbatim}
model.end(res, mf)
\end{verbatim}

\subsubsection{Arguments}

\begin{itemize}
\item {\tt res}: the output from {\tt optim()} or another fitting-algorithm.  
\item {\tt mf}: the model frame output by {\tt model.frame()}.  
\end{itemize}

\subsubsection{Output Values}
A list of regression output, including:
\begin{itemize}
\item {\tt coefficients}: the optimized parameters.
\item {\tt variance}: the variance-covariance matrix (the negative
  inverse of the Hessian matrix returned from the optimization
  procedure).  
\item {\tt terms}:  the terms object.  See {\tt help(terms.object)}
  for more information.
\item {\tt \dots}: additional elements passed from {\tt res}.
\end{itemize}

\subsubsection{See Also}
\begin{itemize}
\item \Sref{s:new} for an overview of how to write a new model.
\end{itemize}

\subsubsection{Contributors}

Kosuke Imai, Gary King, Olivia Lau, and Ferdinand Alimadhi.

%%% Local Variables: 
%%% mode: latex
%%% TeX-master: "~/zelig/docs/zelig"
%%% End: 
