\section{{\tt put.start}: Set specific starting values for certain parameters}
\label{put.start}

\subsubsection{Description}

After calling {\tt set.start()} to create default starting values, use
{\tt put.start()} to change starting values for specific parameters or
parameter sets. 

\subsubsection{Syntax}
\begin{verbatim}
put.start(start.val, value, terms, eqn)
\end{verbatim}

\subsubsection{Arguments}

\begin{itemize}
\item {\tt start.val}: the vector of starting values created by {\tt
set.start()}. 
\item {\tt value}: the scalar or vector of replacement starting
values.  
\item {\tt terms}: the terms output from {\tt model.frame.multiple()}.
\item {\tt eqn}: the parameters for which you would like to replace
the default values with {\tt value}.    
\end{itemize}

\subsubsection{Output Values}
A vector of starting values (of the same length as {\tt start.val}).  

\subsubsection{See Also}
\begin{itemize}
\item \Sref{set.start} to set default starting values.  
\item \Sref{s:new} for an overview of the procedure to add
  models to Zelig.
\end{itemize}

\subsubsection{Contributors}

Kosuke Imai, Gary King, Olivia Lau, and Ferdinand Alimadhi.

%%% Local Variables: 
%%% mode: latex
%%% TeX-master: "~/zelig/docs/zelig"
%%% End: 
