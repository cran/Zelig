\subsection{ \texttt{network}: Format Individual Matricies into a Data Frame for Social Network Analysis }\label{network}

\subsubsection{Description}
This function accepts individual matricies as its inputs, combining
the input matricies into a single data frame which can then
be used in the {\tt data} argument for social network analysis (models
\verb|"netlm"| and \verb|"netlogit"|) in Zelig.\\

\subsubsection{Usage}
\begin{verbatim}
network(...)
\end{verbatim}

\subsubsection{Arguments}

\begin{enumerate}
\item[...] Sociomatricies. These can be given as named arguments and should be given in the order the in which the user wishes them to appear in the dataframe.
\end{enumerate}

\subsubsection{Details}
The \verb|network()| function creates a dataframe which contains
matricies instead of vectors as its variables.  Inputs to the function
should all be square matricies and can be given as named arguments. 

\noindent A matrix within the data frame can be called as a named
argument with 
\begin{verbatim}
> data.name$matrix.name 
\end{verbatim} %$ 
or numerically with 
\begin{verbatim}
> data.name[i]
\end{verbatim}


\subsubsection{Example}

Attach sample data (individual matricies from the {\tt sna.ex} dataset):
\begin{verbatim}
data(Var1, Var2, Var3, Var4, Var5)
\end{verbatim} 

To make a dataframe formated for social network analysis:
\begin{verbatim}
> friendship <- network(Var1, Var2, Var3, Var4, Var5)
\end{verbatim}

\subsubsection{Contributors}
This function was written by Skyler J. Cranmer. 
