\section{{\tt print}: Printing Quantities of Interest}
\label{ss:print}

\subsubsection{Description}
The {\tt print} command formats summary output and allows users to
specify the number of decimal places as an optional argument.  

\subsubsection{Syntax}
\begin{verbatim}
> print(x, digits = 3, print.x = FALSE)
\end{verbatim}

\subsubsection{Arguments}

\begin{itemize}
\item {\tt x}: the object to be printed may be {\tt z.out} output
  from {\tt zelig()}, {\tt x.out} output from {\tt setx()}, {\tt s.out}
  output from {\tt sim()}, or other R data structures.
\item {\tt digits}: the minimum number of significant digits to return
  for all elements of $x < 0$.  By default, {\tt print()} avoids
  scientific notation, but setting the number of digits to 1 will
  frequently force output in scientific notation.  The number of {\tt
    digits} is not the number of significant digits for all output
  values, but the minimum number of significant digits for the
  smallest value in {\tt x} between -1 and 1; this governs the number
  of significant digits in the rest of the values with decimal output.
\item {\tt print.x}: a logical value for {\tt sim()} output, which
  specifies whether to print a summary ({\tt print.x = FALSE}, the
  default) of the {\tt x} and {\tt x1} \emph{inputs} to {\tt sim()},
  or the complete set of inputs (optionally, {\tt print.x = TRUE}).
\end{itemize}

\subsubsection{Examples}
\begin{verbatim}
> print(summary(z.out), digits = 2)
> print(summary(s.out), digits = 3, print.x = TRUE)
\end{verbatim}

\subsubsection{See Also}

Advanced users may wish to refer to {\tt help(print)}.

\subsubsection{Contributors}

Kosuke Imai, Gary King, and Olivia Lau added print methods for {\tt
  sim()} output, and {\tt summary()} output for Zelig objects.   

%%% Local Variables: 
%%% mode: latex
%%% TeX-master: t
%%% End: 
