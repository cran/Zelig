\section{{\tt parse.formula}: Parsing the inputs}
\label{parse.formula}

\subsubsection{Description}
Parse the input formula (or list of formulas) into the standard format
described below.  Since labels for this format will vary by model,
{\tt parse.formula()} will evaluate a function {\tt describe.model()},
where {\tt model} is given as an input to {\tt parse.formula()}.

If the {\tt check.model()} function has more than one parameter for
which {\tt ExpVar = TRUE} and {\tt DepVar = TRUE}, then the
user-specified equations must have labels to match those parameters,
else {\tt parse.formula()} should return an error. In addition, if the
formula entries are not unambiguous, then {\tt parse.formula()} should
return an error.
  
\subsubsection{Syntax}
\begin{verbatim}
> fml <- parse.formula(formula, model, data = NULL)
\end{verbatim}

\subsubsection{Arguments}
\begin{itemize}
  \item {\tt formula}: either a single formula or a list of formula
objects.  
  \item {\tt model}: a character string specifying the name of the model.
  \item {\tt data}: an optional data frame for models that require a
factor response variable.
\end{itemize}

\subsubsection{Output Values}
The output is a list of formula objects with class {\tt ("multiple",
"list")}.  Let's say that the name of the model is {\tt
"bivariate.probit"}, and the corresponding describe function is {\tt
describe.bivariate.probit()}, which identifies {\tt mu1} and {\tt mu2}
as systematic components, and an ancillary parameter {\tt rho}, which
may be parameterized, but is estimated as a scalar by default.  Given
this model, Table \ref{good.formulas} gives acceptable user inputs.  

\subsubsection{Examples}
\begin{verbatim}
formulae <- list(cbind(import, export) ~ coop + cost + target)
fml <- parse.formula(formulae, model = "bivariate.probit")
D <- model.frame(fml, data = mydata)
\end{verbatim}

\subsubsection{See Also}
\begin{itemize}
  \item \Sref{ui} for commented examples of how {\tt parse.formula()}
and {\tt describe.model()} work together.  
  \item \Sref{tag} for constraints between coefficients in a multiple
equation context.  
\end{itemize}

\subsubsection{Contributors}

Kosuke Imai, Gary King, Olivia Lau, and Ferdinand Alimadhi.

\clearpage
\begin{landscape}
\begin{table}[h!]
\caption{Examples of acceptable short-hand for user-specified
formulas, using bivariate probit as an example}
\label{good.formulas}
\begin{center}
\begin{tabular}{lll}
\\
 & User Input    & Output from {\tt parse.formula()} \\
\hline
\\
Same covariates, & \verb|cbind(y1, y2) ~ x1 + x2 * x3|  & \verb|list(mu1 = y1 ~ x1 + x2 * x3,|\\
separate effects & & \verb|     mu2 = y2 ~ x1 + x2 * x3,| \\
& &                  \verb|     rho = ~ 1)| \\
\\
With $\rho$ as a & \verb|list(cbind(y1, y2) ~ x1 + x2,| &
\verb|list(mu1 = y1 ~ x1 + x2,|\\
systematic equation & \verb|    rho = ~ x4 + x5)| & \verb|     mu2 = y2 ~ x1 + x2,|\\
& & \verb|     rho = ~ x4 + x5)| \\
\\
With constraints & \verb|list(mu1 = y1 ~ x1 + tag(x2, "x2"),| &
\verb|list(mu1 = y1 ~ x1 + tag(x2, "x2"),| \\ 
(same variable)  & \verb|     mu2 = y2 ~ x3 + tag(x2, "x2"))| &
\verb|     mu2 = y2 ~ x3 + tag(x2, "x2"),|\\
& & \verb|     rho = ~ 1)|\\
\\
With constraints & \verb|list(mu1 = y1 ~ x1 + tag(x2, "z1"),| &
\verb|list(mu1 = y1 ~ x1 + tag(x2, "z1"),| \\ 
(different variables)  & \verb|     mu2 = y2 ~ x3 + tag(x4, "z1"))| &
\verb|     mu2 = y2 ~ x3 + tag(x4, "z1"),|\\
& & \verb|     rho = ~ 1)|\\
\end{tabular}
\end{center}
\end{table}
\end{landscape}



%%% Local Variables: 
%%% mode: latex
%%% TeX-master: t
%%% End: 












