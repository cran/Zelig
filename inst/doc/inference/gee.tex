\subsection{General Estimating Equation}

Consider a logitudinal data set where $Y_{it}$ denotes the scalar
outcome variable and $X_{it}$ represents an $(k \times 1)$ vector of
explanatory variables for each individual $i$ ($i=1,\dots,N$) at each
time period $t$ ($t=1,\dots,T$).  The stochastic component is
given by
\begin{eqnarray*}
  Y_{it} & \sim & f(y_{it}\mid \mu_{it}, \sigma^2_{it})
\end{eqnarray*}
where $y_{it}$ is a realization of $Y_{it}$, $\mu_{it}$ is the mean,
and $\sigma^2_{it}$ is the variance of $Y_{it}$.  Then the two
systematic components model the mean and variance as follows
\begin{eqnarray*}
\mu_{it} & \equiv & E(Y_{it}\mid X_i) \; = \; g^{-1}(X_i^\top \beta), \\
\sigma^2_{it} & \equiv & {\rm Var}(Y_{it}\mid \mu_{it}) \; = \; \phi\, v(\mu_{it}).
\end{eqnarray*}
where $\beta$ is the vector of coefficients, and $\phi$ is a
dispersion parameter.  The fucntions $g(\cdot)$ and $v(\cdot)$ are
known, and are called the ``link function'' and ``variance function,''
respectively.  GEE models may permit a third systematic component that
allows the covariance to vary over individuals as
\begin{eqnarray*}
  {\rm Cov}(Y_i) = A_i^{\frac{1}{2}} R(\alpha) A_i^{\frac{1}{2}},
\end{eqnarray*}
where $R(\alpha)$ is the $(T \times T)$ correlation matrix (all
diagonal elements equal to 1) and is a function of a parameter
$\alpha$, and $A_i$ is a diagonal matrix of variances
$\sigma^2_{it}=\phi v(\mu_{it})$. There are various ways to specify
the correlation structure:
\begin{itemize}
\item Independence (\texttt{constr = "independence"}): ${\rm
    Cor}(Y_{it}, Y_{it'})=0$ for all $t$ and $t'$ with $t\ne t'$.
\item Exchangeable (\texttt{constr = "exchangeable"}): ${\rm
    Cor}(Y_{it}, Y_{it'})=\alpha$ for all $t$ and $t'$ with $t\ne t'$.
\item $m$th order autoregressive (\texttt{constr = "AR-M"} and
  \texttt{Mv} must be set to $m$): For example, the first order
  autoregressive model implies ${\rm Cor}(Y_{it},
  Y_{it'})=\alpha^{|t-t'|}$ for all $t$ and $t'$ with $t\ne t'$.
\item Stationary $m$ dependent (\texttt{constr = "stat\_M\_dep"} and
  \texttt{Mv} must be set to $m$}):
  $${\rm Cor}(Y_{it}, Y_{it'})=\left\{\begin{array}{ccc}
      \alpha_{|t_j-t_j'|} & {\rm if} & |t_j-t_j'|\le m \\ 0 & {\rm if}
      & |t_j-t_j'| > m
    \end{array}\right.$$
\item Non-stationary $m$ dependent (\texttt{constr =
    "non\_stat\_M\_dep"} and \texttt{Mv} must be set to $m$): {\bf I'm
    not sure what this is!}
\item Unstructured (\texttt{constr = "unstructured"}): ${\rm
    Cor}(Y_{it}, Y_{it'})=\alpha_{tt'}$ for all $t$ and $t'$ with
  $t\ne t'$.
\end{itemize}

Then, the GEE estimator is a solution to the following estimating
equation,
\begin{eqnarray*}
\sum_{i=1}^N D_i^\top V_i (Y_i - \mu_i) & = & 0
\end{eqnarray*}
where $Y_i=(Y_{i1}, Y_{i2}, \dots, Y_{iN})$ and $\mu_i=(\mu_{i1},
\mu_{i2}, \dots,\mu_{iN})$ are vectors of length $T$, and $D_i = d
\mu_i/ d\beta$.  The consistency of the GEE estimator for $\beta$ does
not depend on the specification of the correlation matrix. That is,
even with a misspecified correlation matrix, the estimate of $\beta$
is consistent. However, the correct correlation matrix specification
will make estimation efficient, and the closer the specified
correlation structure is to the true structure, the more efficient is
the estimator.  GEE also utilizes the sandwich estimator to produce
robust standard errors, which produces consistent standard errors even
if the correlation and variance specifications are incorrect.



%%% Local Variables: 
%%% mode: latex
%%% TeX-master: "~/research/zelig/docs/zelig"
%%% End: 
