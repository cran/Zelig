\section{{\tt mloglm}: Multinomial Log-Linear Regression
for Contingency Table Models}\label{mloglm}

Log-linear models are for modeling contingency tables, the
cross-tabulation of discrete individual-level variables.  Contingency
table models take as the ``unit of analysis'' for the purpose of the
statistical procedure, the cell of a contingency table.  The
``dependent variable'' is then the count within each cell, and the
explanatory variables indicate what categories the cells fall into.
These models are highly efficient computationally since there are so
few ``observations,'' but they are asymptotically equivalent to
logistic regression models run on the unpacked individual level data.

\subsubsection{Syntax}

\begin{verbatim}
> estimate <- zelig(Y ~ X1 + X2, model = "mloglm", data = mydata)
> Xval <- setx(estimate)
> results <- sim(estimate, x = Xval)
\end{verbatim}

\subsubsection{Examples}

\subsubsection{Model}

\subsubsection{Quantities of Interest}

\subsubsection{Output Values}

The output of each Zelig command contains useful information which you
may view.  For example, if you run \texttt{estimate <- zelig(y \~\,
  x, model = "mloglm", data)}, then you may examine the available
information in \texttt{estimate} by using \texttt{names(estimate)},
see the {\tt coefficients} by using {\tt estimate\$coefficients}, and
a default summary of information through \texttt{summary(estimate)}.
Other elements available through the {\tt \$} operator are listed
below.

\begin{itemize}
\item From the {\tt zelig()} output stored in {\tt estimate}, you may extract:
   \begin{itemize}
   \item {\tt coefficients}: parameter estimates for the explanatory
     variables.
   \item {\tt deviance}: the residual deviance.
   \item {\tt fitted.values}: the $n \times m$ matrix of in-sample
     fitted values.
   \item {\tt df.residual}: the residual degrees of freedom.
   \item {\tt edf}: the effective degrees of freedom.  
   \item {\tt AIC}: Akaike's An Information Criterion (minus twice the
     maximized log-likelihood plus twice the number of coefficients).
   \item {\tt Hessian}: the Hessian matrix.
   \end{itemize}

\item From {\tt summary(estimate)}, you may extract: 
   \begin{itemize}
   \item {\tt coefficients}: the parameter estimates with their
     associated standard errors, $p$-values, and $t$-statistics.
     covariances.  
   \end{itemize}

\item From the {\tt sim()} output stored in {\tt results}:
   \begin{itemize}
   \item {\tt qi\$ev}: the simulated expected (or fitted values) for
     the specified values of {\tt x}.  
   \item {\tt qi\$rd}: the difference in the expected values (or first
     difference) for the values specified in {\tt x} and
     {\tt x1}.
   \end{itemize}
\end{itemize}

\subsubsection{Contributors}

The multinomial logit model is part of the nnet library by Brian D.
Ripley.  Please cite the model as:
\begin{verse}
\bibentry{Ripley1996}.
\end{verse}

Advanced users may wish to refer to the R-help for
\texttt{help(multinom)} and 
\begin{verse}
\bibentry{VenRip2002}.
\end{verse}

Kosuke Imai, Gary King, and Olivia Lau added Zelig functionality.  

%%% Local Variables: 
%%% mode: latex
%%% TeX-master: t
%%% TeX-master: t
%%% End: 














