\section{{\tt bprobit}: Bivariate Probit Regression for Two
Dichotomous Dependent Variables}\label{bprobit}

Use the bivariate probit regression model if you have two binary
dependent variables $(Y_1, Y_2)$, and wish to model them jointly as a
function of some explanatory variables.  Each pair of dependent
variables $(Y_{i1}, Y_{i2})$ has four potential outcomes, $(Y_{i1}=1,
Y_{i2}=1)$, $(Y_{i1}=1, Y_{i2}=0)$, $(Y_{i1}=0, Y_{i2}=1)$, and
$(Y_{i1}=0, Y_{i2}=0)$.  The joint probability for each of these four
outcomes is modeled with three systematic components: the marginal
Pr$(Y_{i1} = 1)$ and Pr$(Y_{i2} = 1)$, and the correlation parameter
$\rho$ for the two marginal distributions.  Each of these systematic
components may be modeled as functions of (possibly different) sets of
explanatory variables.

\subsubsection{Syntax}

\begin{verbatim}
> z.out <- zelig(list(mu1 = Y1 ~ X1 + X2, 
                      mu2 = Y2 ~ X1 + X3,
                      rho = ~ 1),
                 model = "bprobit", data = mydata)
> x.out <- setx(z.out)
> s.out <- sim(z.out, x = x.out)
\end{verbatim}

\subsubsection{Input Values}

In every bivariate probit specification, there are three equations
which correspond to each dependent variable ($Y_1$, $Y_2$), and the
correlation parameter $\rho$.  Since the correlation parameter does
not correspond to one of the dependent variables, the model estimates
$\rho$ as a constant by default.  Hence, only two formulas (for
$\mu_1$ and $\mu_2$) are required.  If the explanatory variables for
$\mu_1$ and $\mu_2$ are the same and effects are estimated separately
for each parameter, you may use the following short hand:  
\begin{verbatim}
fml <- list(cbind(Y1,Y2) ~ X1 + X2)
\end{verbatim}
which has the same meaning as:  
\begin{verbatim}
fml <- list(mu1 = Y1 ~ X1 + X2,  
            mu2 = Y2 ~ X1 + X2, 
            rho = ~ 1)
\end{verbatim}
You may use the function {\tt tag()} to constrain variables across
equations.  The {\tt tag()} function takes a variable and a label for
the effect parameter.  Below, the constrained effect of {\tt
x3} in both equations is called the {\tt age} parameter:  
\begin{verbatim}
fml <- list(mu1 = y1 ~ x1 + tag(x3, "age"), 
            mu2 = y2 ~ x2 + tag(x3, "age"))
\end{verbatim}
You may also constrain different variables across different equations
to have the same effect.  

\subsubsection{Examples}

\begin{enumerate}

\item {Basic Example} \label{basic.bp}

Load the data and estimate the model:  
\begin{verbatim}
> data(sanction)
> z.out1 <- zelig(cbind(import, export) ~ coop + cost + target, 
                  model = "bprobit", data = sanction)
\end{verbatim}
By default, {\tt zelig()} estimates two effect parameters
for each explanatory variable in addition to the correlation coefficient;
this formulation is parametrically independent (estimating
unconstrained effects for each explanatory variable), but
stochastically dependent because the models share a correlation parameter.
\newline \newline Generate baseline values for the explanatory
variables (with cost set to 1, net gain to sender) and alternative
values (with cost set to 4, major loss to sender):
\begin{verbatim}
> x.low <- setx(z.out1, cost = 1)
> x.high <- setx(z.out1, cost = 4)
\end{verbatim}
Simulate fitted values and first differences:  
\begin{verbatim}
> s.out1 <- sim(z.out1, x = x.low, x1 = x.high)
> summary(s.out1)
> plot(s.out1)
\end{verbatim}

\item {Joint Estimation of a Model with Different Sets of Explanatory Variables}\label{sto.dep.probit}

Load the sample data:
\begin{verbatim}
> data(sanction)
\end{verbatim}
Estimate the statistical model, with {\tt import} a function of {\tt
  coop} in the first equation and {\tt export} a function of {\tt
  cost} and {\tt target} in the second equation:
\begin{verbatim}
> fml2 <- list(mu1 = import ~ coop, 
               mu2 = export ~ cost + target)
> z.out2 <- zelig(fml2, model = "bprobit", data = sanction)
> summary(z.out2)
\end{verbatim}
Set the explanatory variables to their means:
\begin{verbatim}
> x.out2 <- setx(z.out2)
\end{verbatim}
Simulate draws from the posterior distribution:
\begin{verbatim}
> s.out2 <- sim(z.out2, x = x.out2)
> summary(s.out2)
> plot(s.out2)
\end{verbatim}

\item Joint Estimation of a Parametrically and Stochastically
Dependent Model 
\label{pdep.p}
  
Load the sample data:  
\begin{verbatim}
> data(sanction)
\end{verbatim}
A bivariate model is parametrically dependent if $Y_1$ and $Y_2$ share
some or all explanatory variables, {\it and} the effects of the shared
explanatory variables are jointly estimated.  For example,
\begin{verbatim}
> fml3 <- list(mu1 = import ~ tag(coop,"coop") + tag(cost,"cost") + 
                          tag(target,"target"), 
               mu2 = export ~ tag(coop,"coop") + tag(cost,"cost") + 
                          tag(target,"target"))
> z.out3 <- zelig(fml3, model = "bprobit", data = sanction)
> summary(z.out3)
\end{verbatim}
Note that this model only returns one parameter estimate for each of
{\tt coop}, {\tt cost}, and {\tt target}.  Contrast this to
Example~\ref{basic.bp} which returns two parameter estimates for each
of the explanatory variables.  \newline \newline Set values for the
explanatory variables:
\begin{verbatim}
> x.out3 <- setx(z.out3, cost = 1:4)
\end{verbatim} 
Draw simulated expected values:  
\begin{verbatim}
> s.out3 <- sim(z.out3, x = x.out3)
> summary(s.out3)
\end{verbatim}

\end{enumerate}

\subsubsection{Model}

For each observation, define two binary dependent variables, $Y_1$ and
$Y_2$, each of which take the value of either 0 or 1 (in the
following, we suppress the observation index $i$).  We model the joint
outcome $(Y_1$, $Y_2)$ using two marginal probabilities for each
dependent variable, and the correlation parameter, which describes how
the two dependent variables are related. 
%Define $Y_{rs}$ such that it
%is equal to 1 when $Y_1=r$ and $Y_2=s$ and is 0 otherwise where $r$
%and $s$ take a value of either 0 or 1. Then, the model is defined as
%follows,

\begin{itemize}
\item The \emph{stochastic component} is described by two latent (unobserved)
  continuous variables which follow the bivariate Normal distribution:
\begin{eqnarray*}
  \left ( \begin{array}{c} 
      Y_1^* \\
      Y_2^* 
    \end{array}
  \right ) &\sim &  
  N_2 \left \{ \left ( 
      \begin{array}{c}
        \mu_1 \\ \mu_2
      \end{array} \right ), \left( \begin{array}{cc}
                 1 & \rho \\
                 \rho & 1 
                 \end{array} \right) \right\},
\end{eqnarray*}
where $\mu_j$ is a mean for $Y_j^*$ and $\rho$ is a scalar correlation
parameter. The following observation mechanism links the observed
dependent variables, $Y_j$, with these latent variables
\begin{eqnarray*}
Y_j & = & \left \{ \begin{array}{cc}
                   1 & {\rm if} \; Y_j^* \ge 0, \\
                   0 & {\rm otherwise.}
                   \end{array} 
                   \right.
\end{eqnarray*}

%Alternatively, the stochastic component for the observed dependent
%variables can be written as
%\begin{eqnarray*}
%  Y_{11} &\sim& \textrm{Bernoulli}(y_{11} \mid \pi_{11}) \\
%  Y_{10} &\sim& \textrm{Bernoulli}(y_{10} \mid \pi_{10}) \\
% Y_{01} &\sim& \textrm{Bernoulli}(y_{01} \mid \pi_{01})
%\end{eqnarray*}
%where $\pi_{rs}=\Pr(Y_1=r, Y_2=s)$ is the joint probability, and
%$\pi_{00}=1-\pi_{11}-\pi_{10}-\pi_{01}$. Each of these joint
%probabilities is modeled using the bivariate normal cumulative
%distribution function.

\item The \emph{systemic components} for each observation are 
  \begin{eqnarray*}
    \mu_j & = & x_{j} \beta_j \quad {\rm for} \quad j=1,2, \\
    \rho & = & \frac{\exp(x_3 \beta_3) - 1}{\exp(x_3 \beta_3) + 1}.
\end{eqnarray*}

\end{itemize}

\subsubsection{Quantities of Interest}
For $n$ simulations, expected values form an $n \times 4$
matrix.  
\begin{itemize}
\item The expected values ({\tt qi\$ev}) for the binomial probit model
  are the predicted joint probabilities. Simulations of $\beta_1$,
  $\beta_2$, and $\beta_3$ (drawn form their sampling distributions)
  are substituted into the systematic components, to find simulations
  of the predicted joint probabilities $\pi_{rs}=\Pr(Y_1=r, Y_2=s)$:
\begin{eqnarray*}
\pi_{11} &= \Pr(Y_1^* \geq 0 , Y_2^* \geq 0) &= \int_0^{\infty}
\int_0^{\infty} \phi_2 (\mu_1, \mu_2, \rho) \, dY_2^*\, dY_1^* \\
\pi_{10} &= \Pr(Y_1^* \geq 0 , Y_2^* < 0)  &= \int_0^{\infty}
\int_{-\infty}^{0} \phi_2 (\mu_1, \mu_2, \rho) \, dY_2^*\, dY_1^*\\
\pi_{01} &= \Pr(Y_1^* < 0 , Y_2^* \geq 0)  &= \int_{-\infty}^{0}
\int_0^{\infty} \phi_2 (\mu_1, \mu_2, \rho) \, dY_2^*\, dY_1^*\\
\pi_{11} &= \Pr(Y_1^* < 0 , Y_2^* < 0)  &= \int_{-\infty}^{0}
\int_{-\infty}^{0} \phi_2 (\mu_1, \mu_2, \rho) \, dY_2^*\, dY_1^*\\
\end{eqnarray*}
where $r$ and $s$ may take a value of either 0 or 1, $\phi_2$ is the
bivariate Normal density.
  
\item The predicted values ({\tt qi\$pr}) are draws from the
  multinomial distribution given the expected joint probabilities.  

\item The first difference ({\tt qi\$fd}) in each of the predicted joint
  probabilities are given by
  $$\textrm{FD}_{rs} = \Pr(Y_1=r, Y_2=s \mid x_1)-\Pr(Y_1=r, Y_2=s
  \mid x).$$
  
\item The risk ratio ({\tt qi\$rr}) for each of the predicted joint
  probabilities are given by
\begin{equation*}
\textrm{RR}_{rs} = \frac{\Pr(Y_1=r, Y_2=s \mid x_1)}{\Pr(Y_1=r, Y_2=s \mid x)}.
\end{equation*}

\item In conditional prediction models, the average expected treatment
  effect ({\tt att.ev}) for the treatment group is 
    \begin{equation*} \frac{1}{\sum_{i=1}^n t_i}\sum_{i:t_i=1}^n \left\{ Y_{ij}(t_i=1) -
      E[Y_{ij}(t_i=0)] \right\} \textrm{ for } j = 1,2,
    \end{equation*} 
    where $t_i$ is a binary explanatory variable defining the treatment
    ($t_i=1$) and control ($t_i=0$) groups.  Variation in the
    simulations are due to uncertainty in simulating $E[Y_{ij}(t_i=0)]$,
    the counterfactual expected value of $Y_{ij}$ for observations in the
    treatment group, under the assumption that everything stays the
    same except that the treatment indicator is switched to $t_i=0$.

\item In conditional prediction models, the average predicted treatment
  effect ({\tt att.pr}) for the treatment group is 
    \begin{equation*} \frac{1}{\sum_{i=1}^n t_i}\sum_{i:t_i=1}^n \left\{ Y_{ij}(t_i=1) -
      \widehat{Y_{ij}(t_i=0)}\right\} \textrm{ for } j = 1,2,
    \end{equation*} 
    where $t_i$ is a binary explanatory variable defining the treatment
    ($t_i=1$) and control ($t_i=0$) groups.  Variation in the
    simulations are due to uncertainty in simulating
    $\widehat{Y_{ij}(t_i=0)}$, the counterfactual predicted value of
    $Y_{ij}$ for observations in the treatment group, under the
    assumption that everything stays the same except that the
    treatment indicator is switched to $t_i=0$.

\end{itemize}

\subsubsection{Output Values}

The output of each Zelig command contains useful information which you
may view.  For example, if you run \texttt{z.out <- zelig(y \~\, x,
  model = "bprobit", data)}, then you may examine the available
information in \texttt{z.out} by using \texttt{names(z.out)},
see the {\tt coefficients} by using {\tt z.out\$coefficients}, and
obtain a default summary of information through
\texttt{summary(z.out)}.  Other elements available through the {\tt
  \$} operator are listed below.

\begin{itemize}
\item From the {\tt zelig()} output object {\tt z.out}, you may
  extract:
   \begin{itemize}
   \item {\tt coefficients}: the named vector of coefficients.   
   \item {\tt fitted.values}: an $n \times 4$ matrix of the in-sample
     fitted values.
   \item {\tt predictors}: an $n \times 3$ matrix of the linear
     predictors $x_j \beta_j$.
   \item {\tt residuals}: an $n \times 3$ matrix of the residuals.  
   \item {\tt df.residual}: the residual degrees of freedom.  
   \item {\tt df.total}: the total degrees of freedom.
   \item {\tt rss}: the residual sum of squares.  
   \item {\tt y}: an $n \times 2$ matrix of the dependent variables.  
   \item {\tt zelig.data}: the input data frame if {\tt save.data = TRUE}.  
   \end{itemize}

\item From {\tt summary(z.out)}, you may extract:
\begin{itemize}
  \item {\tt coef3}: a table of the coefficients with their associated
    standard errors and $t$-statistics.
  \item {\tt cov.unscaled}: the variance-covariance matrix. 
  \item {\tt pearson.resid}: an $n \times 3$ matrix of the Pearson residuals.  
\end{itemize}

\item From the {\tt sim()} output object {\tt s.out}, you may extract
  quantities of interest arranged as arrays indexed by simulation
  $\times$ quantity $\times$ {\tt x}-observation (for more than one
  {\tt x}-observation; otherwise the quantities are matrices).  Available quantities
  are:  

   \begin{itemize}
   \item {\tt qi\$ev}: the simulated expected values (joint predicted
     probabilities) for the specified values of {\tt x}.
   \item {\tt qi\$pr}: the simulated predicted outcomes drawn from a
     distribution defined by the joint predicted probabilities.
   \item {\tt qi\$fd}: the simulated first difference in the predicted
     probabilities for the values specified in {\tt x} and {\tt x1}.
   \item {\tt qi\$rr}: the simulated risk ratio in the predicted
     probabilities for given {\tt x} and {\tt x1}.
   \item {\tt qi\$att.ev}: the simulated average expected treatment
     effect for the treated from conditional prediction models.  
   \item {\tt qi\$att.pr}: the simulated average predicted treatment
     effect for the treated from conditional prediction models.  
   \end{itemize}
\end{itemize}

\subsubsection{Contributors}

The bivariate probit function is part of the VGAM package by Thomas Yee.
Please cite this model as:
\begin{verse}
\bibentry{YeeHas03}.
\end{verse}

In addition, advanced users may wish to refer to \texttt{help(vglm)}
in the VGAM package.  Additional documentation is available at
\hlink{http://www.stat.auckland.ac.nz/\~\,yee}{http://www.stat.auckland.ac.nz/~yee}.

Sample data are from 
\begin{verse}
\bibentry{Martin92}.
\end{verse}
Kosuke Imai, Gary King, and Olivia Lau added Zelig functionality.  

%%% Local Variables: 
%%% mode: latex
%%% TeX-master: t
%%% End: 





