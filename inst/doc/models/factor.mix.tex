\section{\texttt{factor.mix}: Mixed Data Factor Analysis}\label{factor.mix}

Mixed data factor analysis takes both continuous and ordinal dependent
variables and estimates a model for a given number of latent factors.
The model is estimated using a Markov Chain Monte Carlo algorithm
(Gibbs sampler with data augmentation).  Alternative models include
Bayesian factor analysis for continuous variables
(\Sref{factor.bayes}) and Bayesian factor analysis for ordinal
variables (\Sref{factor.ord}).  

\subsubsection{Syntax}
\begin{verbatim}
> z.out <- zelig(cbind(Y1 ,Y2, Y3) ~ NULL, factors = 1, 
                model = "factor.mix", data = mydata)
\end{verbatim}

\subsubsection{Inputs}
{\tt zelig()} accepts the following arguments for {\tt factor.mix}:  
\begin{itemize}

\item \texttt{Y1}, {\tt Y2}, \texttt{Y3}, {\tt \dots}: The dependent variables of
interest, which can be a mix of ordinal and continuous variables.  You
must have more dependent variables than factors.

\item \texttt{factors}: The number of the factors to be fitted.

\end{itemize}


\subsubsection{Additional Inputs}

The model accepts the following additional arguments to monitor
convergence:  
\begin{itemize}
\item \texttt{lambda.constraints}: A list that contains the equality or 
inequality constraints on the factor loadings. 

\begin{itemize}

\item {\tt varname = list()}: by default, no constraints are
imposed.

\item \texttt{varname = list(d, c)}: constrains the
$d$th loading for the variable named \texttt{varname} to be equal to \texttt{c}.

\item \texttt{varname = list(d, "+")}: constrains the
$d$th loading for the variable named \texttt{varname} to be positive;

\item \texttt{varname = list(d, "-")}: constrains the
$d$th loading for the variable named \texttt{varname} to be negative.
\end{itemize} 
Unlike Bayesian factor analysis for continuous variables
(\Sref{factor.bayes}), the first column of $\Lambda$ corresponds to
negative item difficulty parameters and should not be constrained in
general.

\item \texttt{std.mean}: defaults to {\tt TRUE}, which rescales the
continuous manifest variables to have mean 0.  

\item \texttt{std.var}: defaults to {\tt TRUE}. which rescales the
continuous manifest variables to have unit variance.  
\end{itemize}


\noindent \texttt{factor.mix} accepts the following additional arguments 
to monitor the sampling scheme for the Markov chain:

\begin{itemize}
\item \texttt{burnin}: number of the initial MCMC iterations to be 
 discarded. The default value is 1,000.

\item \texttt{mcmc}: number of the MCMC iterations after burnin.
 The default value is 20,000.

\item \texttt{thin}: thinning interval for the Markov chain. Only every 
 \texttt{thin}-th draw from the Markov chain is kept. The value of 
\texttt{mcmc} must be divisible by this value. The default value is 1.

\item \texttt{tune}: tuning parameter, which can be either a
scalar or a vector of length $K$. The value of the tuning parameter
must be positive. The default value is \texttt{1.2}.

\item \texttt{verbose}: defaults to {\tt FALSE}. If \texttt{TRUE}, the progress 
 of the sampler (every $10\%$) is printed to the screen. The default 
is \texttt{FALSE}.

\item \texttt{seed}: seed for the random number generator. The default
is \texttt{NA} which corresponds to a random seed 12345.

\item \texttt{lambda.start}: starting values of the factor loading
matrix $\Lambda$ for the Markov chain, either a scalar (starting
values of the unconstrained loadings will be set to that value), or a
matrix with compatible dimensions.  The default is \texttt{NA}, where
the start values for the first column of $\Lambda$ are set based on
the observed pattern, while for the rest of the columns of $\Lambda$,
the start values are set to be 0 for unconstrained factor loadings,
and 1 or $-$1 for constrained factor loadings (depending on the nature
of the constraints).

\item \texttt{psi.start}: starting values for the diagonals of the error variance
(uniquenesses) matrix. Since the starting values for the ordinal
variables are constrained to 1 (to identify the model), you may only
specify the starting values for the continuous variables.  For the
continuous variables, you may specify {\tt psi.start} as a scalar or a
vector with length equal to the number of continuous variables.  If a
scalar, that starting value is recycled for all continuous variables.
If a vector, the starting values should correspond to each of the
continuous variables.  The default value is \texttt{NA}, which means
the starting values of all the continuous variable uniqueness are set
to 0.5.

\item \texttt{store.lambda}: defaults to {\tt TRUE}, storing the
posterior draws of the factor loadings.  

\item \texttt{store.scores}: defaults to {\tt FALSE}.  If {\tt TRUE},
the posterior draws of the factor scores are stored. (Storing factor
scores may take large amount of memory for a a large number of draws
or observations.)

\end{itemize}
Use the following additional arguments to specify prior parameters used in the model:
\begin{itemize}

\item \texttt{l0}: mean of the Normal prior for the factor
loadings, either a scalar or a matrix with the same dimensions as
$\Lambda$.  If a scalar value, then that value will be the prior mean
for all the factor loadings. The default value is 0.

\item \texttt{L0}: precision parameter of Normal prior for the factor
loadings, either a scalar or a matrix with the same dimensions as $\Lambda$.  
If a scalar value, then the precision matrix will be 
a diagonal matrix with the diagonal elements set to that value. 
The default value is 0 which leads to an improper prior.

\item \texttt{a0}: {\tt a0/2} is the shape parameter of the Inverse Gamma priors for 
the uniquenesses. It can take a scalar value or a vector. The default
value is 0.001.

\item \texttt{b0}: {\tt b0/2} is the shape parameter of the Inverse Gamma priors for 
the uniquenesses. It can take a scalar value or a vector. The default
value is 0.001.

\end{itemize}
Zelig users may wish to refer to \texttt{help(MCMCmixfactanal)} for more 
information.

\subsubsection{Convergence}

Users should verify that the Markov Chain converges to its stationary 
distribution.  After running the \texttt{zelig()} function but before 
performing \texttt{setx()}, users may conduct the following 
convergence diagnostics tests:

\begin{itemize}
\item \texttt{geweke.diag(z.out\$coefficients)}: The Geweke diagnostic tests
the null hypothesis that the Markov chain is in the stationary distribution
and produces z-statistics for each estimated parameter. 
 
\item \texttt{heidel.diag(z.out\$coefficients)}: The Heidelberger-Welch
diagnostic first tests the null hypothesis that the Markov Chain is in the
stationary distribution and produces p-values for each estimated parameter. 
 Calling \texttt{heidel.diag()} also produces
output that indicates whether the mean of a marginal posterior distribution
can be estimated with sufficient precision, assuming that the Markov Chain is
in the stationary distribution.

\item \texttt{raftery.diag(z.out\$coefficients)}: The Raftery diagnostic
indicates how long the Markov Chain should run before considering draws from
the marginal posterior distributions sufficiently representative of the
stationary distribution.
\end{itemize}

\noindent If there is evidence of non-convergence, adjust the values 
for \texttt{burnin} and \texttt{mcmc} and rerun \texttt{zelig()}.

Advanced users may wish to refer to \texttt{help(geweke.diag)}, 
\texttt{help(heidel.diag)}, and \texttt{help(raftery.diag)} for more 
information about these diagnostics. 


\subsubsection{Examples}

\begin{enumerate}
\item {Basic Example} \\
Attaching the sample  dataset:
\begin{verbatim}
> data(PErisk)
\end{verbatim}
Factor analysis for mixed data using \texttt{factor.mix}:
\begin{verbatim}
> z.out<-zelig(cbind(courts, barb2, prsexp2, prscorr2, gdpw2) ~ NULL, 
               data = PErisk, model = "factor.mix", factors = 1,  
               burnin = 5000, mcmc = 100000, thin = 50, verbose = TRUE, 
               L0 = 0.25)
\end{verbatim}
Checking for convergence before summarizing the estimates:
\begin{verbatim}
> geweke.diag(z.out$coefficients)
> heidel.diag(z.out$coefficients)
> summary(z.out)
\end{verbatim}

\end{enumerate}

\subsubsection{Model}

Let $Y_i$ be a $K$-vector of observed variables for observation $i$, 
The $k$th variable can be either continuous or ordinal. When $Y_{ik}$ is an
ordinal variable, it takes value from 1 to $J_k$ for $k=1,\ldots, K$ and for 
$i =1, \ldots, n$. The distribution of $Y_{ik}$ is assumed to be 
governed by another $K$-vector of unobserved continuous variable $Y_{ik}^*$. 
There are $d$ underlying factors. When $Y_{ik}$ is continuous, we let 
$Y_{ik}^*=Y_{ik}$.

\begin{itemize}
\item The \emph{stochastic component} is described in terms of $Y_i^*$:
\begin{eqnarray*}
Y_{i}^*  &\sim& \textrm{Normal}_K (\mu_i, I_K),
\end{eqnarray*}
where $Y_i^*=(Y_{i1}^*, \ldots, Y_{iK}^*)$, and $\mu_i=(\mu_{i1},\ldots, \mu_{iK})$.

For ordinal $Y_{ik}$,
\begin{eqnarray*}
Y_{ik} = j \quad {\rm if} \quad \gamma_{(j-1),k} \le Y_{ik}^* \le
\gamma_{jk} \quad \textrm{ for } \quad j=1,\ldots, J_k; k=1,\ldots, K.
\end{eqnarray*}
where $\gamma_{jk}, j=0,\ldots, J$ are the threshold parameters for the $k$th
variable with the following constraints, $\gamma_{lk} < \gamma_{mk}$ for $l < m$, and $\gamma_{0k}=-\infty, \gamma_{J_k k}=\infty$ for any $k=1, \ldots, K$.
It follows that the probability of observing $Y_{ik}$ belonging to category 
$j$ is,
\begin{eqnarray*}
\Pr(Y_{ik}=j) &=&\Phi(\gamma_{jk} \mid \mu_{ik})-\Phi(\gamma_{(j-1),k} \mid \mu_{ik}) \quad 
\textrm{ for } j=1,\ldots,J_k
\end{eqnarray*}
where $\Phi(\cdot\mid\mu_{ik})$ is the cumulative distribution function of the 
Normal distribution with mean $\mu_{ik}$ and variance 1.

\item The \emph{systematic component} is given by,
\begin{eqnarray*}
\mu_i &=& \Lambda\phi_i,
\end{eqnarray*}
where $\Lambda$ is a $K \times d$ matrix of factor loadings for each variable,
$\phi_i$ is a $d$-vector of factor scores for observation $i$. Note both 
$\Lambda$ and $\phi$ are estimated.. 

\item The independent conjugate \emph{prior} for each $\Lambda_{ij}$  is given by
\begin{eqnarray*}
\Lambda_{ij} &\sim& \textrm{Normal}(l_{0_{ij}}, L_{0_{ij}}^{-1}) 
\textrm{ for } i=1,\ldots, k; \quad j=1,\ldots, d.
\end{eqnarray*}

\item The \emph{prior} for $\phi_i$ is,
\begin{eqnarray*}
\phi_{i} \sim \textrm{Normal}(0, I_{d-1}), \quad {\rm for} \quad i=2, \ldots, n.
\end{eqnarray*}
where $I_{d-1}$ is a $ (d-1) \times (d-1) $ identity matrix. Note the
first element of $\phi_i$ is 1. 
\end{itemize}


\subsubsection{Output Values}

The output of each Zelig command contains useful information which you may
view. For example, if you run:
\begin{verbatim}
z.out <- zelig(cbind(Y1, Y2, Y3), model = "factor.mix", data)
\end{verbatim}

\noindent then you may examine the available information in \texttt{z.out} by
using \texttt{names(z.out)}, see the draws from the posterior distribution of
the \texttt{coefficients} by using \texttt{z.out\$coefficients}, and view a default
summary of information through \texttt{summary(z.out)}. Other elements
available through the \texttt{\$} operator are listed below.

\begin{itemize}
\item From the \texttt{zelig()} output object \texttt{z.out}, you may extract:

\begin{itemize}
\item \texttt{coefficients}: draws from the posterior distributions
of the estimated factor loadings, the estimated cut points $\gamma$ for each
variable. Note the first element of $\gamma$ is normalized to be 0. If 
\texttt{store.scores = TRUE}, the estimated factors scores are also contained in 
\texttt{coefficients}.

   \item {\tt zelig.data}: the input data frame if {\tt save.data = TRUE}.  
\item \texttt{seed}: the random seed used in the model.   

\end{itemize}

\item Since there are no explanatory variables, the \texttt{sim()} procedure is
not applicable for factor analysis models.

\end{itemize}

\subsubsection{Contributors}
Factor analysis for mixed dependent variables function is part of the MCMCpack library by Andrew D.
Martin and Kevin M. Quinn. If you use this model, please cite:
\begin{verse}
\bibentry{MarQui05}.
\end{verse}
The convergence diagnostics are part of the CODA library
by Martyn Plummer, Nicky Best, Kate Cowles, and Karen Vines. These diagnostics
should be cited as: 
\begin{verse}
\bibentry{PluBesCowVin05}.
\end{verse}
\noindent Sample data are adapted from 
\begin{verse}
\bibentry{MarQui05}.
\end{verse}


\noindent Ben Goodrich and Ying Lu enabled \texttt{factor.mix} to work with Zelig.

%%% Local Variables:
%%% mode: latex
%%% TeX-master: t
%%% End:

