\section{{\tt netlogit}: Network Logistic Regression for
Dichotomous Proximity Matrix Dependent Variables}\label{netlogit} 

Use network logistic squares regression analysis for a dependent
variable that is a binary-valued proximity matrix (a.k.a.\
sociomatrices, adjacency matrices, or matrix representations of
directed graphs).

\subsubsection{Syntax}
\begin{verbatim}
> z.out <- zelig(y ~ x1 + x2, model = "netlogit", data = mydata) 
> x.out <- setx(z.out)
> s.out <- sim(z.out, x = x.out)
\end{verbatim}

\subsubsection{Examples}
\begin{enumerate}
\item Basic Example

Load the sample data and format it for social network analysis:
\begin{verbatim} 
data(friendship)
\end{verbatim}
Estimate model:
\begin{verbatim}
> z.out <- zelig(friends ~ advice + prestige + perpower, 
                 model = "netlogit", data = friendship)
> summary(z.out)
\end{verbatim}
Setting values for the explanatory variables to their default values:
\begin{verbatim}
x.out <- setx(z.out)
\end{verbatim}
Simulating quantities of interest from the sampling distribution.
\begin{verbatim}
> s.out <- sim(z.out, x = x.out) 
> summary(s.out) 
> plot(s.out) 
\end{verbatim}

\item Simulating First Differences

Estimating the risk difference (and risk ratio) between low personal
power (25th percentile) and high education (75th percentile) while all
the other variables are held at their default values.

\begin{verbatim}
> x.high <- setx(z.out, perpower = quantile(friendship$perpower, prob = 0.75))    
> x.low  <- setx(z.out, educate = quantile(friendship$perpower, prob = 0.25))
> s.out2 <- sim(z.out, x = x.high, x1 = x.low)   
> summary(s.out2)   
> plot(s.out2)   
\end{verbatim}
\end{enumerate}

\subsubsection{Model}
The {\tt netlogit} model performs a logistic regression of the
sociomatrix $\mathbf{Y}$, a $m \times m$ matrix representing network
ties, on a set of sociomatrices $\mathbf{X}$. This network regression
model is a directly analogue to standard logistic regression
element-wise on the appropriately vectorized matrices. Sociomatrices
are vectorized by creating $Y$, an $m^{2} \times 1$ vector to
represent the sociomatrix. The vectorization which produces the $Y$
vector from the $\mathbf{Y}$ matrix is preformed by simple
row-concatenation of $\mathbf{Y}$. For example if $\mathbf{Y}$ is a
$15 \times 15$ matrix, the $\mathbf{Y}_{1,1}$ element is the first
element of $Y$, and the $\mathbf{Y}_{21}$ element is the second
element of $Y$ and so on. Once the input matrices are vectorized,
standard logistic regression is performed.

Let $Y_{i}$ be the binary dependent variable, produced by vectorizing
a binary sociomatrix, for observation $i$ which takes the value of
either 0 or 1.
\begin{itemize}
\item The \emph{stochastic component} is given by 
\begin{eqnarray*}
Y_{i} & \sim \text{Bernoulli} (y_{i} | \pi_{i})\\
& = \pi_{i}^{y_{i}} (1 - \pi_{i})^{1 - y_{i}}
\end{eqnarray*}
where $\pi_{i} = \text{Pr}(Y_{i} = 1)$.
\item The \emph{systematic component} is given by:
\begin{equation*}
\pi_{i} = \frac{1}{1 + \exp(-x_{i}\beta)}.
\end{equation*}
where $x_{i}$ is the vector of $k$ covariates for observation $i$ and
$\beta$ is the vector of coefficients.
\end{itemize}

\subsubsection{Quantities of Interest}
The quantities of interest for the network logistic regression are the
same as those for the standard logistic regression.
\begin{itemize}
\item The expected values ({\tt qi\$ev}) for the netlogit model are
simulations of the predicted probability of a success:   
\begin{equation*}
E(Y) = \pi_{i} = \frac{1}{1 + \exp(-x_{i}\beta)}.
\end{equation*}
given draws of $\beta$ from its sampling distribution.

\item The predicted values ({\tt qi\$pr}) are draws from the Binomial
distribution with mean equal to the simulated expected value
$\pi_{i}$. 

\item The first difference ({\tt qi\$fd}) for the logit model is defined as 
\begin{equation*}
FD = \text{Pr}(Y = 1 | x_{1}) - \text{Pr}(Y = 1| x)
\end{equation*}
\end{itemize}


\subsubsection{Output Values}

The output of each Zelig command contains useful information which you
may view. For example, you run {\tt z.out <- zelig(y ~ x, model
="netlogit", data)}, then you may examine the available information in
{\tt z.out} by using {\tt names(z.out)}, see the coefficients by using
{\tt z.out\$coefficients}, and a default summary of information
through {\tt summary(z.out)}. Other elements available through the
{\tt \$} operator are listed below.
\begin{itemize} 
\item From the {\tt zelig()} output stored in {\tt z.out}, you may extract:
\begin{itemize}
\item {\tt coefficients}: parameter estimates for the explanatory variables.
\item {\tt fitted.values}: the vector of fitted values for the
explanatory variables. 
\item {\tt residuals}: the working residuals in the final iteration of
the IWLS fit.  
\item {\tt linear.predictors}: the vector of $x_{i}\beta$.
\item {\tt aic}: Akaike's Information Criterion (minus twice the
maximized log-likelihood plus twice the number of coefficients). 
\item {\tt bic}: the Bayesian Information Criterion (minus twice the
maximized log-likelihood plus the number of coefficients times log
$n$). 
\item {\tt df.residual}: the residual degrees of freedom.
\item {\tt df.null}: the residual degrees of freedom for the null model. 
\item {\tt zelig.data}: the input data frame if {\tt save.data = TRUE} 
\end{itemize} 
\item From {\tt summary(z.out)}, you may extract:
\begin{itemize}
\item {\tt mod.coefficients}: the parameter estimates with their associated standard errors, $p$-values, and $t$ statistics. 
\item {\tt cov.scaled}: a $k \times k$ matrix of scaled covariances.
\item {\tt cov.unscaled}: a $k \times k$ matrix of unscaled
covariances. 
\item {\tt ctable}: a $2 \times 2$ table of predicted vs.\ actual
values.  Each cell tabulates whether the predicted value $\widehat{Y}
\in \{0,1\}$ corresponds to the observed $Y \in \{0,1\}$. 
\end{itemize}
\item From the {\tt sim()} output stored in {\tt s.out}, you may extract:
\begin{itemize}
\item {\tt qi\$ev}: the simulated expected probabilities for the
specified values of {\tt x}.
\item {\tt qi\$pr}: the simulated predicted values for the specified values of {\tt x}.
\item {\tt qi\$fd}: the simulated first differences in the expected
probabilities simulated from {\tt x} and {\tt x1}.
\end{itemize}
\end{itemize}

\subsubsection{Contributors}
The network logistic regression is part of the sna package by Carter
T. Butts.  Please cite the model as \\
\begin{verse}
\bibentry{ButCar01}.
\end{verse}
In addition, advanced users may wish to refer to {\tt help(netlogit)}.
Sample data are fictional. Skyler J.\ Cranmer added Zelig functionality. 
