\documentclass[oneside,letterpaper,12pt]{book}
\usepackage{Rd}
\usepackage{/usr/lib64/R/share/texmf/Sweave}
\usepackage{bibentry}
\usepackage{upquote}
\usepackage{graphicx}
\usepackage{natbib}
\usepackage[reqno]{amsmath}
\usepackage{amssymb}
\usepackage{amsfonts}
\usepackage{amsmath}
\usepackage{verbatim}
\usepackage{epsf}
\usepackage{url}
\usepackage{html}
\usepackage{dcolumn}
\usepackage{multirow}
\usepackage{fullpage}
\usepackage{lscape}
\usepackage[all]{xy}
% \usepackage[pdftex, bookmarksopen=true,bookmarksnumbered=true,
%   linkcolor=webred]{hyperref}
\bibpunct{(}{)}{;}{a}{}{,}
\newcolumntype{.}{D{.}{.}{-1}}
\newcolumntype{d}[1]{D{.}{.}{#1}}
\htmladdtonavigation{
  \htmladdnormallink{%
    \htmladdimg{http://gking.harvard.edu/pics/home.gif}}
  {http://gking.harvard.edu/}}
\newcommand{\MatchIt}{{\sc MatchIt}}
\newcommand{\hlink}{\htmladdnormallink}
\newcommand{\Sref}[1]{Section~\ref{#1}}
\newcommand{\fullrvers}{2.5.1}
\newcommand{\rvers}{2.5}
\newcommand{\rwvers}{R-2.5.1}
%\renewcommand{\bibentry}{\citealt}

\bodytext{ BACKGROUND="http://gking.harvard.edu/pics/temple.jpg"}
\setcounter{tocdepth}{2}
 \begin{document}\section{{\tt newpainters}: The Discretized Painter's Data of de Piles}\label{ss:newpainters}
\keyword{datasets}{newpainters}
\begin{Description}\relax
The original painters data contain the subjective assessment, 
on a 0 to 20 integer scale, of 54 classical painters. The
newpainters data discretizes the subjective assessment by
quartiles with thresholds 25\%, 50\%, 75\%. The painters were 
assessed on four characteristics: composition, drawing, 
colour and expression.  The data is due to the Eighteenth century 
art critic, de Piles.
\end{Description}
\begin{Usage}
\begin{verbatim}data(newpainters)\end{verbatim}
\end{Usage}
\begin{Format}\relax
A table containing 5 variables ("Composition", "Drawing", "Colour", 
"Expression", and "School") and 54 observations.
\end{Format}
\begin{Source}\relax
A. J. Weekes (1986).``A Genstat Primer''. Edward Arnold.

M. Davenport and G. Studdert-Kennedy (1972). ``The statistical
analysis of aesthetic judgement: an exploration.'' \emph{Applied
Statistics}, vol. 21,  pp. 324--333.

I. T. Jolliffe (1986) ``Principal Component Analysis.'' Springer.
\end{Source}
\begin{References}\relax
Venables, W. N. and Ripley, B. D. (2002) ``Modern Applied
Statistics with S,'' Fourth edition.  Springer.
\end{References}

\end{document}
