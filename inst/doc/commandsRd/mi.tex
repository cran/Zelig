\documentclass[oneside,letterpaper,12pt]{book}
\usepackage{Rd}
\usepackage{/usr/lib64/R/share/texmf/Sweave}
\usepackage{bibentry}
\usepackage{upquote}
\usepackage{graphicx}
\usepackage{natbib}
\usepackage[reqno]{amsmath}
\usepackage{amssymb}
\usepackage{amsfonts}
\usepackage{amsmath}
\usepackage{verbatim}
\usepackage{epsf}
\usepackage{url}
\usepackage{html}
\usepackage{dcolumn}
\usepackage{multirow}
\usepackage{fullpage}
\usepackage{lscape}
\usepackage[all]{xy}
% \usepackage[pdftex, bookmarksopen=true,bookmarksnumbered=true,
%   linkcolor=webred]{hyperref}
\bibpunct{(}{)}{;}{a}{}{,}
\newcolumntype{.}{D{.}{.}{-1}}
\newcolumntype{d}[1]{D{.}{.}{#1}}
\htmladdtonavigation{
  \htmladdnormallink{%
    \htmladdimg{http://gking.harvard.edu/pics/home.gif}}
  {http://gking.harvard.edu/}}
\newcommand{\MatchIt}{{\sc MatchIt}}
\newcommand{\hlink}{\htmladdnormallink}
\newcommand{\Sref}[1]{Section~\ref{#1}}
\newcommand{\fullrvers}{2.5.1}
\newcommand{\rvers}{2.5}
\newcommand{\rwvers}{R-2.5.1}
%\renewcommand{\bibentry}{\citealt}

\bodytext{ BACKGROUND="http://gking.harvard.edu/pics/temple.jpg"}
\setcounter{tocdepth}{2}
 \begin{document}\section{{\tt mi}: Bundle multiply imputed data sets as a list}\label{ss:mi}
\keyword{methods}{mi}
\begin{Description}\relax
The code \code{mi} bundles multiply imputed data sets as a
list for further analysis.
\end{Description}
\begin{Usage}
\begin{verbatim}
  mi(...)
\end{verbatim}
\end{Usage}
\begin{Arguments}
\begin{ldescription}
\item[\code{...}] multiply imputed data sets, separated by commas. The
arguments can be tagged by \code{name=data} where \code{name} is the
element named used for the data set \code{data}.
\end{ldescription}
\end{Arguments}
\begin{Value}
The list containing each multiply imputed data set as an
element. The class name is \code{mi}. The list can be inputted into
\code{zelig} for statistical analysis with multiply imputed data
sets. See \code{zelig} for details.
\end{Value}
\begin{Author}\relax
Kosuke Imai <\email{kimai@princeton.edu}>; Gary King
<\email{king@harvard.edu}>; Olivia Lau <\email{olau@fas.harvard.edu}>
\end{Author}
\begin{SeeAlso}\relax
The full Zelig manual is available at
\url{http://gking.harvard.edu/zelig}.
\end{SeeAlso}
\begin{Examples}
\begin{ExampleCode}
  data(immi1, immi2, immi3, immi4, immi5)
  mi(immi1, immi2, immi3, immi4, immi5)
\end{ExampleCode}
\end{Examples}

\end{document}
