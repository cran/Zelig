\documentclass[oneside,letterpaper,12pt]{book}
\usepackage{Rd}
\usepackage{/usr/lib64/R/share/texmf/Sweave}
\usepackage{bibentry}
\usepackage{upquote}
\usepackage{graphicx}
\usepackage{natbib}
\usepackage[reqno]{amsmath}
\usepackage{amssymb}
\usepackage{amsfonts}
\usepackage{amsmath}
\usepackage{verbatim}
\usepackage{epsf}
\usepackage{url}
\usepackage{html}
\usepackage{dcolumn}
\usepackage{multirow}
\usepackage{fullpage}
\usepackage{lscape}
\usepackage[all]{xy}
% \usepackage[pdftex, bookmarksopen=true,bookmarksnumbered=true,
%   linkcolor=webred]{hyperref}
\bibpunct{(}{)}{;}{a}{}{,}
\newcolumntype{.}{D{.}{.}{-1}}
\newcolumntype{d}[1]{D{.}{.}{#1}}
\htmladdtonavigation{
  \htmladdnormallink{%
    \htmladdimg{http://gking.harvard.edu/pics/home.gif}}
  {http://gking.harvard.edu/}}
\newcommand{\MatchIt}{{\sc MatchIt}}
\newcommand{\hlink}{\htmladdnormallink}
\newcommand{\Sref}[1]{Section~\ref{#1}}
\newcommand{\fullrvers}{2.5.1}
\newcommand{\rvers}{2.5}
\newcommand{\rwvers}{R-2.5.1}
%\renewcommand{\bibentry}{\citealt}

\bodytext{ BACKGROUND="http://gking.harvard.edu/pics/temple.jpg"}
\setcounter{tocdepth}{2}
 \begin{document}\section{{\tt match.data}: Output matched data sets}\label{ss:match.data}
\keyword{methods}{match.data}
\begin{Description}\relax
The code \code{match.data} creates output data sets from the \code{matchit}
matching algorithm.
\end{Description}
\begin{Usage}
\begin{verbatim}
match.data <- match.data(object, group = "all")
\end{verbatim}
\end{Usage}
\begin{Arguments}
\begin{ldescription}
\item[\code{object}] Stored output from \code{matchit}.
\item[\code{group}] Which units to output.  Selecting "all" (default) gives all
matched units (treated and control), "treat" gives just the matched
treated units, and "control" gives just the matched control units.
\end{ldescription}
\end{Arguments}
\begin{Value}
The \code{match.data} command generates a matched data set from
the output of the \code{matchit} function, according to the options
selected in the \code{group} argument.  The matched data set contains
the additional variables: 
\begin{ldescription}
\item[\code{pscore}] The propensity score for each unit.
\item[\code{psclass}] The subclass index for each unit (if applicable).
\item[\code{psweights}] The weight for each unit (generated from the matching
procedure).
\end{ldescription}

See the \code{matchit} documentation for more details on these items.
\end{Value}
\begin{Author}\relax
Daniel Ho <\email{deho@fas.harvard.edu}>; Kosuke Imai
<\email{kimai@princeton.edu}>; Gary King
<\email{king@harvard.edu}>; Elizabeth Stuart<\email{stuart@stat.harvard.edu}>
\end{Author}
\begin{SeeAlso}\relax
The complete documentation for \code{matchit} is available online at
\url{http://gking.harvard.edu/matchit}.
\end{SeeAlso}

\end{document}
