\documentclass[oneside,letterpaper,12pt]{book}
\usepackage{Rd}
\usepackage{/usr/lib64/R/share/texmf/Sweave}
\usepackage{bibentry}
\usepackage{upquote}
\usepackage{graphicx}
\usepackage{natbib}
\usepackage[reqno]{amsmath}
\usepackage{amssymb}
\usepackage{amsfonts}
\usepackage{amsmath}
\usepackage{verbatim}
\usepackage{epsf}
\usepackage{url}
\usepackage{html}
\usepackage{dcolumn}
\usepackage{multirow}
\usepackage{fullpage}
\usepackage{lscape}
\usepackage[all]{xy}
% \usepackage[pdftex, bookmarksopen=true,bookmarksnumbered=true,
%   linkcolor=webred]{hyperref}
\bibpunct{(}{)}{;}{a}{}{,}
\newcolumntype{.}{D{.}{.}{-1}}
\newcolumntype{d}[1]{D{.}{.}{#1}}
\htmladdtonavigation{
  \htmladdnormallink{%
    \htmladdimg{http://gking.harvard.edu/pics/home.gif}}
  {http://gking.harvard.edu/}}
\newcommand{\MatchIt}{{\sc MatchIt}}
\newcommand{\hlink}{\htmladdnormallink}
\newcommand{\Sref}[1]{Section~\ref{#1}}
\newcommand{\fullrvers}{2.5.1}
\newcommand{\rvers}{2.5}
\newcommand{\rwvers}{R-2.5.1}
%\renewcommand{\bibentry}{\citealt}

\bodytext{ BACKGROUND="http://gking.harvard.edu/pics/temple.jpg"}
\setcounter{tocdepth}{2}
 \begin{document}\section{{\tt ternaryplot}: Ternary diagram}\label{ss:ternaryplot}
\keyword{hplot}{ternaryplot}
\begin{Description}\relax
Visualizes compositional, 3-dimensional data in an equilateral triangle  
(from the vcd library, Version 0.1-3.3, Date 2004-04-21), using plot graphics.  
Differs from implementation in vcd (0.9-7), which uses grid graphics.
\end{Description}
\begin{Usage}
\begin{verbatim}
ternaryplot(x, scale = 1, dimnames = NULL, dimnames.position = c("corner","edge","none"),
            dimnames.color = "black", id = NULL, id.color = "black", coordinates = FALSE,
            grid = TRUE, grid.color = "gray", labels = c("inside", "outside", "none"),
            labels.color = "darkgray", border = "black", bg = "white", pch = 19, cex = 1,
            prop.size = FALSE, col = "red", main = "ternary plot", ...)
\end{verbatim}
\end{Usage}
\begin{Arguments}
\begin{ldescription}
\item[\code{x}] a matrix with three columns.
\item[\code{scale}] row sums scale to be used.
\item[\code{dimnames}] dimension labels (defaults to the column names of
\code{x}).
\item[\code{dimnames.position, dimnames.color}] position and color of dimension labels.
\item[\code{id}] optional labels to be plotted below the plot
symbols. \code{coordinates} and \code{id} are mutual exclusive.
\item[\code{id.color}] color of these labels.
\item[\code{coordinates}] if \code{TRUE}, the coordinates of the points are
plotted below them. \code{coordinates} and \code{id} are mutual exclusive.
\item[\code{grid}] if \code{TRUE}, a grid is plotted. May optionally
be a string indicating the line type (default: \code{"dotted"}).
\item[\code{grid.color}] grid color.
\item[\code{labels, labels.color}] position and color of the grid labels.
\item[\code{border}] color of the triangle border.
\item[\code{bg}] triangle background.
\item[\code{pch}] plotting character. Defaults to filled dots.
\item[\code{cex}] a numerical value giving the amount by which plotting text
and symbols should be scaled relative to the default. Ignored for
the symbol size if \code{prop.size} is not \code{FALSE}.
\item[\code{prop.size}] if \code{TRUE}, the symbol size is plotted
proportional to the row sum of the three variables, i.e. represents
the weight of the observation.
\item[\code{col}] plotting color.
\item[\code{main}] main title.
\item[\code{...}] additional graphics parameters (see \code{par})
\end{ldescription}
\end{Arguments}
\begin{Details}\relax
A points' coordinates are found by computing the gravity center
of mass points using the data entries as weights. Thus, the coordinates
of a point P(a,b,c), \eqn{a + b + c = 1}{}, are: P(b + c/2, c * sqrt(3)/2).
\end{Details}
\begin{Author}\relax
David Meyer\\
\email{david.meyer@ci.tuwien.ac.at}
\end{Author}
\begin{References}\relax
M. Friendly (2000),
\emph{Visualizing Categorical Data}. SAS Institute, Cary, NC.
\end{References}
\begin{SeeAlso}\relax
\code{\LinkA{ternarypoints}{ternarypoints}}
\end{SeeAlso}
\begin{Examples}
\begin{ExampleCode}
data(mexico)
if (require(VGAM)) { 
z.out <- zelig(as.factor(vote88) ~ pristr + othcok + othsocok, 
                model = "mlogit", data = mexico)
x.out <- setx(z.out)
s.out <- sim(z.out, x = x.out)

ternaryplot(s.out$qi$ev, pch = ".", col = "blue",
            main = "1988 Mexican Presidential Election")
}
\end{ExampleCode}
\end{Examples}

\end{document}
