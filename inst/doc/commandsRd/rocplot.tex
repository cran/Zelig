\documentclass[oneside,letterpaper,12pt]{book}
\usepackage{Rd}
\usepackage{/usr/lib64/R/share/texmf/Sweave}
\usepackage{bibentry}
\usepackage{upquote}
\usepackage{graphicx}
\usepackage{natbib}
\usepackage[reqno]{amsmath}
\usepackage{amssymb}
\usepackage{amsfonts}
\usepackage{amsmath}
\usepackage{verbatim}
\usepackage{epsf}
\usepackage{url}
\usepackage{html}
\usepackage{dcolumn}
\usepackage{multirow}
\usepackage{fullpage}
\usepackage{lscape}
\usepackage[all]{xy}
% \usepackage[pdftex, bookmarksopen=true,bookmarksnumbered=true,
%   linkcolor=webred]{hyperref}
\bibpunct{(}{)}{;}{a}{}{,}
\newcolumntype{.}{D{.}{.}{-1}}
\newcolumntype{d}[1]{D{.}{.}{#1}}
\htmladdtonavigation{
  \htmladdnormallink{%
    \htmladdimg{http://gking.harvard.edu/pics/home.gif}}
  {http://gking.harvard.edu/}}
\newcommand{\MatchIt}{{\sc MatchIt}}
\newcommand{\hlink}{\htmladdnormallink}
\newcommand{\Sref}[1]{Section~\ref{#1}}
\newcommand{\fullrvers}{2.5.1}
\newcommand{\rvers}{2.5}
\newcommand{\rwvers}{R-2.5.1}
%\renewcommand{\bibentry}{\citealt}

\bodytext{ BACKGROUND="http://gking.harvard.edu/pics/temple.jpg"}
\setcounter{tocdepth}{2}
 \begin{document}\section{{\tt rocplot}: Receiver Operator Characteristic Plots}\label{ss:rocplot}
\aliasA{ROC}{rocplot}{ROC}
\aliasA{roc}{rocplot}{roc}
\aliasA{ROCplot}{rocplot}{ROCplot}
\keyword{file}{rocplot}
\begin{Description}\relax
The \code{rocplot} command generates a receiver operator
characteristic plot to compare the in-sample (default) or out-of-sample
fit for two logit or probit regressions.
\end{Description}
\begin{Usage}
\begin{verbatim}
rocplot(y1, y2, fitted1, fitted2, cutoff = seq(from=0, to=1, length=100), 
        lty1 = "solid", lty2 = "dashed", lwd1 = par("lwd"), lwd2 = par("lwd"),
        col1 = par("col"), col2 = par("col"), main, xlab, ylab,
        plot = TRUE, ...)
\end{verbatim}
\end{Usage}
\begin{Arguments}
\begin{ldescription}
\item[\code{y1}] Response variable for the first model.
\item[\code{y2}] Response variable for the second model.
\item[\code{fitted1}] Fitted values for the first model.  These values
may represent either the in-sample or out-of-sample fitted values.
\item[\code{fitted2}] Fitted values for the second model.
\item[\code{cutoff}] A vector of cut-off values between 0 and 1, at
which to evaluate the proportion of 0s and 1s correctly predicted by
the first and second model.  By default, this is 100 increments
between 0 and 1, inclusive.
\item[\code{lty1, lty2}] The line type for the first model (\code{lty1}) and
the second model (\code{lty2}), defaulting to solid and dashed,
respectively.
\item[\code{lwd1, lwd2}] The width of the line for the first model
(\code{lwd1}) and the second model (\code{lwd2}), defaulting to 1 for both.
\item[\code{col1, col2}] The colors of the line for the first
model (\code{col1}) and the second model (\code{col2}), defaulting to
black for both.
\item[\code{main}] a title for the plot.  Defaults to \code{ROC Curve}.
\item[\code{xlab}] a label for the x-axis.  Defaults to \code{Proportion of 1's 
    Correctly Predicted}.
\item[\code{ylab}] a label for the y-axis.  Defaults to \code{Proportion of 0's 
    Correctly Predicted}.
\item[\code{plot}] defaults to \code{TRUE}, which generates a plot to the
selected device.  If \code{FALSE}, returns a list of
items (see below).
\item[\code{...}] Additional parameters passed to plot, including
\code{xlab}, \code{ylab}, and \code{main}.  
\end{ldescription}
\end{Arguments}
\begin{Value}
If \code{plot = TRUE}, \code{rocplot} generates an ROC plot for
two logit or probit models.  If \code{plot = FALSE}, \code{rocplot}
returns a list with the following elements: normal-bracket63bracket-normal
\begin{ldescription}
\item[\code{roc1}] a matrix containing a vector of x-coordinates and
y-coordinates corresponding to the number of ones and zeros correctly
predicted for the first model.
\item[\code{roc2}] a matrix containing a vector of x-coordinates and
y-coordinates corresponding to the number of ones and zeros correctly
predicted for the second model.
\item[\code{area1}] the area under the first ROC curve, calculated using
Reimann sums.
\item[\code{area2}] the area under the second ROC curve, calculated using
Reimann sums.
\end{ldescription}

normal-bracket63bracket-normal
\end{Value}
\begin{Author}\relax
Kosuke Imai <\email{kimai@princeton.edu}>; Gary King
<\email{king@harvard.edu}>; Olivia Lau <\email{olau@fas.harvard.edu}>
\end{Author}
\begin{SeeAlso}\relax
The full Zelig manual (available at
\url{http://gking.harvard.edu/zelig}), \code{plot}, \code{lines}.
\end{SeeAlso}
\begin{Examples}
\begin{ExampleCode}
data(turnout)
z.out1 <- zelig(vote ~ race + educate + age, model = "logit", 
  data = turnout)
z.out2 <- zelig(vote ~ race + educate, model = "logit", 
  data = turnout)
rocplot(z.out1$y, z.out2$y, fitted(z.out1), fitted(z.out2))
\end{ExampleCode}
\end{Examples}

\end{document}
