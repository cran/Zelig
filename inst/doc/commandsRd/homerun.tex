\include{zinput} \begin{document}\section{{\tt homerun}: Sample Data on Home Runs Hit By Mark McGwire and Sammy Sosa in 1998.}\label{ss:homerun}
\keyword{datasets}{homerun}
\begin{Description}\relax
Game-by-game information for the 1998 season for Mark McGwire and Sammy Sosa. Data are a subset of the dataset provided in Simonoff (1998).
\end{Description}
\begin{Usage}
\begin{verbatim}data(homerun)\end{verbatim}
\end{Usage}
\begin{Format}\relax
A data frame containing 5 variables ("gameno", "month", "homeruns", "playerstatus", "player") and 326 observations.  
\describe{
\item[\code{gameno}] an integer variable denoting the game number
\item[\code{month}] a factor variable taking with levels "March" through "September" denoting the month of the game
\item[\code{homeruns}] an integer vector denoting the number of homeruns hit in that game for that player
\item[\code{playerstatus}] an integer vector equal to "0" if the player played in the game, and "1" if they did not.
\item[\code{player}] an  integer vector equal to "0" (McGwire) or "1" (Sosa)
}
\end{Format}
\begin{Source}\relax
\url{http://www.amstat.org}
\end{Source}
\begin{References}\relax
Simonoff, Jeffrey S. 1998. ``Move Over, Roger Maris: Breaking Baseball's Most Famous Record.'' \emph{Journal of Statistics Education} 6(3). Data used are a subset of the data in the article.
\end{References}

\end{document}
