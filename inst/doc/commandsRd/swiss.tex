\documentclass[oneside,letterpaper,12pt]{book}
\usepackage{Rd}
\usepackage{/usr/lib64/R/share/texmf/Sweave}
\usepackage{bibentry}
\usepackage{upquote}
\usepackage{graphicx}
\usepackage{natbib}
\usepackage[reqno]{amsmath}
\usepackage{amssymb}
\usepackage{amsfonts}
\usepackage{amsmath}
\usepackage{verbatim}
\usepackage{epsf}
\usepackage{url}
\usepackage{html}
\usepackage{dcolumn}
\usepackage{multirow}
\usepackage{fullpage}
\usepackage{lscape}
\usepackage[all]{xy}
% \usepackage[pdftex, bookmarksopen=true,bookmarksnumbered=true,
%   linkcolor=webred]{hyperref}
\bibpunct{(}{)}{;}{a}{}{,}
\newcolumntype{.}{D{.}{.}{-1}}
\newcolumntype{d}[1]{D{.}{.}{#1}}
\htmladdtonavigation{
  \htmladdnormallink{%
    \htmladdimg{http://gking.harvard.edu/pics/home.gif}}
  {http://gking.harvard.edu/}}
\newcommand{\MatchIt}{{\sc MatchIt}}
\newcommand{\hlink}{\htmladdnormallink}
\newcommand{\Sref}[1]{Section~\ref{#1}}
\newcommand{\fullrvers}{2.5.1}
\newcommand{\rvers}{2.5}
\newcommand{\rwvers}{R-2.5.1}
%\renewcommand{\bibentry}{\citealt}

\bodytext{ BACKGROUND="http://gking.harvard.edu/pics/temple.jpg"}
\setcounter{tocdepth}{2}
 \begin{document}\section{{\tt swiss}: Swiss Fertility and Socioeconomic Indicators (1888) Data}\label{ss:swiss}
\keyword{datasets}{swiss}
\begin{Description}\relax
Standardized fertility measure and socio-economic indicators for
each of 47 French-speaking provinces of Switzerland at about 1888.
\end{Description}
\begin{Usage}
\begin{verbatim}data(swiss)\end{verbatim}
\end{Usage}
\begin{Format}\relax
A data frame with 47 observations on 6 variables, each of which
is in percent, i.e., in [0,100].

[,1]  Fertility         Ig, "common standardized fertility measure"
[,2]  Agriculture       
[,3]  Examination       
nation
[,4]  Education         
[,5]  Catholic          
[,6]  Infant.Mortality  live births who live less than 1 year.

All variables but 'Fert' give proportions of the population.
\end{Format}
\begin{Source}\relax
Project "16P5", pages 549-551 in

Mosteller, F. and Tukey, J. W. (1977) ``Data Analysis and
Regression: A Second Course in Statistics''. Addison-Wesley,
Reading Mass.

indicating their source as "Data used by permission of Franice van
de Walle. Office of Population Research, Princeton University,
1976.  Unpublished data assembled under NICHD contract number No
1-HD-O-2077."
\end{Source}
\begin{References}\relax
Becker, R. A., Chambers, J. M. and Wilks, A. R. (1988) ``The New S
Language''. Wadsworth \& Brooks/Cole.
\end{References}

\end{document}
