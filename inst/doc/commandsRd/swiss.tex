\include{zinput} \begin{document}\section{{\tt swiss}: Swiss Fertility and Socioeconomic Indicators (1888) Data}\label{ss:swiss}
\keyword{datasets}{swiss}
\begin{Description}\relax
Standardized fertility measure and socio-economic indicators for
each of 47 French-speaking provinces of Switzerland at about 1888.
\end{Description}
\begin{Usage}
\begin{verbatim}data(swiss)\end{verbatim}
\end{Usage}
\begin{Format}\relax
A data frame with 47 observations on 6 variables, each of which
is in percent, i.e., in [0,100].

[,1]  Fertility         Ig, "common standardized fertility measure"
[,2]  Agriculture       
[,3]  Examination       
nation
[,4]  Education         
[,5]  Catholic          
[,6]  Infant.Mortality  live births who live less than 1 year.

All variables but 'Fert' give proportions of the population.
\end{Format}
\begin{Source}\relax
Project "16P5", pages 549-551 in

Mosteller, F. and Tukey, J. W. (1977) ``Data Analysis and
Regression: A Second Course in Statistics''. Addison-Wesley,
Reading Mass.

indicating their source as "Data used by permission of Franice van
de Walle. Office of Population Research, Princeton University,
1976.  Unpublished data assembled under NICHD contract number No
1-HD-O-2077."
\end{Source}
\begin{References}\relax
Becker, R. A., Chambers, J. M. and Wilks, A. R. (1988) ``The New S
Language''. Wadsworth \& Brooks/Cole.
\end{References}

\end{document}
