\include{zinput} \begin{document}\section{{\tt summary.zelig}: Summary of Simulated Quantities of Interest}\label{ss:summary.zelig}
\aliasA{summary}{summary.zelig}{summary}
\keyword{file}{summary.zelig}
\begin{Description}\relax
Summarizes the object of class \code{\LinkA{zelig}{zelig}} (output
from \code{\LinkA{sim}{sim}}) which contains simulated quantities of
interst.
\end{Description}
\begin{Usage}
\begin{verbatim}
## S3 method for class 'zelig':
summary(object, subset = NULL, CI = 95, stats = c("mean", "sd"), ...)
\end{verbatim}
\end{Usage}
\begin{Arguments}
\begin{ldescription}
\item[\code{object}] output object from \code{\LinkA{sim}{sim}} (of class
\code{"zelig"}).
\item[\code{subset}] takes one of three values:
\Itemize{
\item[NULL] (default) for more than one observation, summarizes all the
observations at once for each quantity of interest.
\item[a numeric vector] indicates which observations to summarize,
and summarizes each one independently.
\item[all] summarizes all the observations independently.
}

\item[\code{stats}] summary statistics to be calculated.
\item[\code{CI}] a confidence interval to be calculated.
\item[\code{...}] further arguments passed to or from other methods.
\end{ldescription}
\end{Arguments}
\begin{Value}
\begin{ldescription}
\item[\code{sim}] number of simulations, i.e., posterior draws.
\item[\code{x}] values of explanatory variables used for simulation.
\item[\code{x1}] values of explanatory variables used for simulation of first
differences etc.
\item[\code{qi.stats}] summary of quantities of interst.  Use
\code{\LinkA{names}{names}} to view the model-specific items available in
\code{qi.stats}.
\end{ldescription}
\end{Value}
\begin{Author}\relax
Kosuke Imai <\email{kimai@princeton.edu}>; Gary King
<\email{king@harvard.edu}>; Olivia Lau <\email{olau@fas.harvard.edu}>
\end{Author}
\begin{SeeAlso}\relax
\code{\LinkA{zelig}{zelig}}, \code{\LinkA{setx}{setx}}, \code{\LinkA{sim}{sim}},
and \code{\LinkA{names}{names}}, and the full Zelig manual at
\url{http://gking.harvard.edu/zelig}.
\end{SeeAlso}

\end{document}
