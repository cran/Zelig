\documentclass[oneside,letterpaper,12pt]{book}
\usepackage{Rd}
\usepackage{/usr/lib64/R/share/texmf/Sweave}
\usepackage{bibentry}
\usepackage{upquote}
\usepackage{graphicx}
\usepackage{natbib}
\usepackage[reqno]{amsmath}
\usepackage{amssymb}
\usepackage{amsfonts}
\usepackage{amsmath}
\usepackage{verbatim}
\usepackage{epsf}
\usepackage{url}
\usepackage{html}
\usepackage{dcolumn}
\usepackage{multirow}
\usepackage{fullpage}
\usepackage{lscape}
\usepackage[all]{xy}
% \usepackage[pdftex, bookmarksopen=true,bookmarksnumbered=true,
%   linkcolor=webred]{hyperref}
\bibpunct{(}{)}{;}{a}{}{,}
\newcolumntype{.}{D{.}{.}{-1}}
\newcolumntype{d}[1]{D{.}{.}{#1}}
\htmladdtonavigation{
  \htmladdnormallink{%
    \htmladdimg{http://gking.harvard.edu/pics/home.gif}}
  {http://gking.harvard.edu/}}
\newcommand{\MatchIt}{{\sc MatchIt}}
\newcommand{\hlink}{\htmladdnormallink}
\newcommand{\Sref}[1]{Section~\ref{#1}}
\newcommand{\fullrvers}{2.5.1}
\newcommand{\rvers}{2.5}
\newcommand{\rwvers}{R-2.5.1}
%\renewcommand{\bibentry}{\citealt}

\bodytext{ BACKGROUND="http://gking.harvard.edu/pics/temple.jpg"}
\setcounter{tocdepth}{2}
 \begin{document}\section{{\tt PErisk}: Political Economic Risk Data from 62 Countries in 1987}\label{ss:PErisk}
\keyword{datasets}{PErisk}
\begin{Description}\relax
Political Economic Risk Data from 62 Countries in 1987.
\end{Description}
\begin{Usage}
\begin{verbatim}data(PErisk)\end{verbatim}
\end{Usage}
\begin{Format}\relax
A data frame with 62 observations on the following 6 variables.
All data points are from 1987. See Quinn (2004) for more
details. 

country: a factor with levels 'Argentina' 'Australia' 'Austria'
'Bangladesh' 'Belgium' 'Bolivia' 'Botswana' 'Brazil' 'Burma'
'Cameroon' 'Canada' 'Chile' 'Colombia' 'Congo-Kinshasa'
'Costa Rica' 'Cote d'Ivoire' 'Denmark' 'Dominican Republic'
'Ecuador' 'Finland' 'Gambia, The' 'Ghana' 'Greece' 'Hungary'
'India' 'Indonesia' 'Iran' 'Ireland' 'Israel' 'Italy' 'Japan'
'Kenya' 'Korea, South' 'Malawi' 'Malaysia' 'Mexico' 'Morocco'
'New Zealand' 'Nigeria' 'Norway' 'Papua New Guinea'
'Paraguay' 'Philippines' 'Poland' 'Portugal' 'Sierra Leone'
'Singapore' 'South Africa' 'Spain' 'Sri Lanka' 'Sweden'
'Switzerland' 'Syria' 'Thailand' 'Togo' 'Tunisia' 'Turkey'
'United Kingdom' 'Uruguay' 'Venezuela' 'Zambia' 'Zimbabwe'

courts: an ordered factor with levels '0' < '1'.'courts' is an
indicator of whether the country in question is judged to
have an independent judiciary. From Henisz (2002).

barb2: a numeric vector giving the natural log of the black market
premium in each country. The black market premium is coded as
the black market exchange rate (local currency per dollar)
divided by the official exchange  rate minus 1. From
Marshall, Gurr, and Harff (2002). 

prsexp2: an ordered factor with levels '0' < '1' < '2' < '3' < '4'
< '5', giving the lack of expropriation risk. From Marshall,
Gurr, and Harff (2002).

prscorr2: an ordered factor with levels '0' < '1' < '2' < '3' < '4'
< '5', measuring the lack of corruption. From Marshall, Gurr,
and Harff (2002).

gdpw2: a numeric vector giving the natural log of real GDP per
worker in 1985 international prices. From Alvarez et al.
(1999).
\end{Format}
\begin{Source}\relax
Mike Alvarez, Jose Antonio Cheibub, Fernando Limongi, and Adam
Przeworski. 1999. ``ACLP Political and Economic Database.'' <URL:
http://www.ssc.upenn.edu/~cheibub/data/>.

Witold J. Henisz. 2002. ``The Political Constraint Index (POLCON)
Dataset.'' \ <URL:
http://www-management.wharton.upenn.edu/henisz/POLCON/ContactInfo.
html>.

Monty G. Marshall, Ted Robert Gurr, and Barbara Harff. 2002.
``State Failure Task Force Problem Set.'' <URL:
http://www.cidcm.umd.edu/inscr/stfail/index.htm>.
\end{Source}
\begin{References}\relax
Kevin M. Quinn. 2004. ``Bayesian Factor Analysis for Mixed Ordinal
and Continuous Response.'' \emph{Political Analyis}. Vol. 12, pp.338--353.
\end{References}

\end{document}
