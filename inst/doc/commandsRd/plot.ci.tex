\documentclass[oneside,letterpaper,12pt]{book}
\usepackage{Rd}
\usepackage{/usr/lib64/R/share/texmf/Sweave}
\usepackage{bibentry}
\usepackage{upquote}
\usepackage{graphicx}
\usepackage{natbib}
\usepackage[reqno]{amsmath}
\usepackage{amssymb}
\usepackage{amsfonts}
\usepackage{amsmath}
\usepackage{verbatim}
\usepackage{epsf}
\usepackage{url}
\usepackage{html}
\usepackage{dcolumn}
\usepackage{multirow}
\usepackage{fullpage}
\usepackage{lscape}
\usepackage[all]{xy}
% \usepackage[pdftex, bookmarksopen=true,bookmarksnumbered=true,
%   linkcolor=webred]{hyperref}
\bibpunct{(}{)}{;}{a}{}{,}
\newcolumntype{.}{D{.}{.}{-1}}
\newcolumntype{d}[1]{D{.}{.}{#1}}
\htmladdtonavigation{
  \htmladdnormallink{%
    \htmladdimg{http://gking.harvard.edu/pics/home.gif}}
  {http://gking.harvard.edu/}}
\newcommand{\MatchIt}{{\sc MatchIt}}
\newcommand{\hlink}{\htmladdnormallink}
\newcommand{\Sref}[1]{Section~\ref{#1}}
\newcommand{\fullrvers}{2.5.1}
\newcommand{\rvers}{2.5}
\newcommand{\rwvers}{R-2.5.1}
%\renewcommand{\bibentry}{\citealt}

\bodytext{ BACKGROUND="http://gking.harvard.edu/pics/temple.jpg"}
\setcounter{tocdepth}{2}
 \begin{document}\section{{\tt plot.ci}: Plotting Vertical confidence Intervals}\label{ss:plot.ci}
\keyword{hplot}{plot.ci}
\begin{Description}\relax
The \code{plot.ci} command generates vertical
confidence intervals for linear or generalized linear univariate
response models.
\end{Description}
\begin{Usage}
\begin{verbatim}
plot.ci(x, CI = 95, qi = "ev", main = "", ylab = NULL, xlab = NULL,
        xlim = NULL, ylim = NULL, col = c("red", "blue"), ...) 
\end{verbatim}
\end{Usage}
\begin{Arguments}
\begin{ldescription}
\item[\code{x}] stored output from \code{sim}.  The \code{x\$x} and optional
\code{x\$x1} values used to generate the \code{sim} output object must
have more than one observation.
\item[\code{CI}] the selected confidence interval.  Defaults to 95
percent.
\item[\code{qi}] the selected quantity of interest.  Defaults to
expected values.
\item[\code{main}] a title for the plot.
\item[\code{ylab}] label for the y-axis.
\item[\code{xlab}] label for the x-axis.
\item[\code{xlim}] limits on the x-axis.
\item[\code{ylim}] limits on the y-axis.
\item[\code{col}] a vector of at most two colors for plotting the
expected value given by \code{x} and the alternative set of expected
values given by \code{x1} in \code{sim}.  If the quantity of
interest selected is not the expected value, or \code{x1 = NULL},
only the first color will be used.
\item[\code{...}] Additional parameters passed to \code{plot}.
\end{ldescription}
\end{Arguments}
\begin{Value}
For all univariate response models, \code{plot.ci()} returns vertical
confidence intervals over a specified range of one explanatory
variable.  You may save this plot using the commands described in the
Zelig manual (\url{http://gking.harvard.edu/zelig}).
\end{Value}
\begin{Author}\relax
Kosuke Imai <\email{kimai@princeton.edu}>; Gary King
<\email{king@harvard.edu}>; Olivia Lau <\email{olau@fas.harvard.edu}>
\end{Author}
\begin{SeeAlso}\relax
The full Zelig manual is available at
\url{http://gking.harvard.edu/zelig}, and users may also wish to see
\code{plot}, \code{lines}.
\end{SeeAlso}
\begin{Examples}
\begin{ExampleCode}
data(turnout)
z.out <- zelig(vote ~ race + educate + age + I(age^2) + income,
               model = "logit", data = turnout)
age.range <- 18:95
x.low <- setx(z.out, educate = 12, age = age.range)
x.high <- setx(z.out, educate = 16, age = age.range)
s.out <- sim(z.out, x = x.low, x1 = x.high)
plot.ci(s.out, xlab = "Age in Years",
        ylab = "Predicted Probability of Voting",
        main = "Effect of Education and Age on Voting Behavior")
legend(45, 0.52, legend = c("College Education (16 years)",
       "High School Education (12 years)"), col = c("blue","red"), 
       lty = c("solid"))
\end{ExampleCode}
\end{Examples}

\end{document}
