\include{zinput} \begin{document}\section{{\tt help.zelig}: HTML Help for Zelig Commands and Models}\label{ss:help.zelig}
\keyword{documentation}{help.zelig}
\begin{Description}\relax
The \code{help.zelig} command launches html help for Zelig commands
and supported models.  The full manual is available online at
\url{http://gking.harvard.edu/zelig}.
\end{Description}
\begin{Usage}
\begin{verbatim}
help.zelig(...)
\end{verbatim}
\end{Usage}
\begin{Arguments}
\begin{ldescription}
\item[\code{...}] a Zelig command or model. 
\code{help.zelig(command)} will take you to an index of Zelig
commands and \code{help.zelig(model)} will take you to a list of
models. 
\end{ldescription}
\end{Arguments}
\begin{Author}\relax
Kosuke Imai <\email{kimai@princeton.edu}>; Gary King
<\email{king@harvard.edu}>; Olivia Lau<\email{olau@fas.harvard.edu}>
\end{Author}
\begin{SeeAlso}\relax
The complete document is available online at
\url{http://gking.harvard.edu/zelig}.
\end{SeeAlso}

\end{document}
