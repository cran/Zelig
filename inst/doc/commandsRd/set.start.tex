\documentclass[oneside,letterpaper,12pt]{book}
\usepackage{Rd}
\usepackage{/usr/lib64/R/share/texmf/Sweave}
\usepackage{bibentry}
\usepackage{upquote}
\usepackage{graphicx}
\usepackage{natbib}
\usepackage[reqno]{amsmath}
\usepackage{amssymb}
\usepackage{amsfonts}
\usepackage{amsmath}
\usepackage{verbatim}
\usepackage{epsf}
\usepackage{url}
\usepackage{html}
\usepackage{dcolumn}
\usepackage{multirow}
\usepackage{fullpage}
\usepackage{lscape}
\usepackage[all]{xy}
% \usepackage[pdftex, bookmarksopen=true,bookmarksnumbered=true,
%   linkcolor=webred]{hyperref}
\bibpunct{(}{)}{;}{a}{}{,}
\newcolumntype{.}{D{.}{.}{-1}}
\newcolumntype{d}[1]{D{.}{.}{#1}}
\htmladdtonavigation{
  \htmladdnormallink{%
    \htmladdimg{http://gking.harvard.edu/pics/home.gif}}
  {http://gking.harvard.edu/}}
\newcommand{\MatchIt}{{\sc MatchIt}}
\newcommand{\hlink}{\htmladdnormallink}
\newcommand{\Sref}[1]{Section~\ref{#1}}
\newcommand{\fullrvers}{2.5.1}
\newcommand{\rvers}{2.5}
\newcommand{\rwvers}{R-2.5.1}
%\renewcommand{\bibentry}{\citealt}

\bodytext{ BACKGROUND="http://gking.harvard.edu/pics/temple.jpg"}
\setcounter{tocdepth}{2}
 \begin{document}\section{{\tt set.start}: Set starting values for all parameters}\label{ss:set.start}
\keyword{utilities}{set.start}
\begin{Description}\relax
After using \code{\LinkA{parse.par}{parse.par}} and \code{\LinkA{model.matrix.multiple}{model.matrix.multiple}}, use 
\code{set.start} to set starting values for all parameters.  By default, starting values are set to 0.  If 
you wish to select alternative starting values for certain parameters, use \code{\LinkA{put.start}{put.start}} after 
\code{set.start}.
\end{Description}
\begin{Usage}
\begin{verbatim}
set.start(start.val = NULL, terms)
\end{verbatim}
\end{Usage}
\begin{Arguments}
\begin{ldescription}
\item[\code{start.val}] user-specified starting values.  If \code{NULL} (default), the default 
starting values for all parameters are set to 0.
\item[\code{terms}] the terms output from \code{\LinkA{model.frame.multiple}{model.frame.multiple}}
\end{ldescription}
\end{Arguments}
\begin{Value}
A named vector of starting values for all parameters specified in \code{terms}, defaulting to 0.
\end{Value}
\begin{Author}\relax
Kosuke Imai <\email{kimai@princeton.edu}>; Gary King
<\email{king@harvard.edu}>; Olivia Lau <\email{olau@fas.harvard.edu}>; Ferdinand Alimadhi
<\email{falimadhi@iq.harvard.edu}>
\end{Author}
\begin{SeeAlso}\relax
\code{\LinkA{put.start}{put.start}}, \code{\LinkA{parse.par}{parse.par}}, \code{\LinkA{model.frame.multiple}{model.frame.multiple}}, and the 
full Zelig manual at \url{http://gking.harvard.edu/zelig}.
\end{SeeAlso}
\begin{Examples}
\begin{ExampleCode}
## Not run: 
fml <- parse.formula(formula, model = "bivariate.probit")
D <- model.frame(fml, data = data)
terms <- attr(D, "terms")
start.val <- set.start(start.val = NULL, terms)
## End(Not run)\end{ExampleCode}
\end{Examples}

\end{document}
