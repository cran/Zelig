\documentclass[oneside,letterpaper,12pt]{book}
\usepackage{Rd}
\usepackage{/usr/lib64/R/share/texmf/Sweave}
\usepackage{bibentry}
\usepackage{upquote}
\usepackage{graphicx}
\usepackage{natbib}
\usepackage[reqno]{amsmath}
\usepackage{amssymb}
\usepackage{amsfonts}
\usepackage{amsmath}
\usepackage{verbatim}
\usepackage{epsf}
\usepackage{url}
\usepackage{html}
\usepackage{dcolumn}
\usepackage{multirow}
\usepackage{fullpage}
\usepackage{lscape}
\usepackage[all]{xy}
% \usepackage[pdftex, bookmarksopen=true,bookmarksnumbered=true,
%   linkcolor=webred]{hyperref}
\bibpunct{(}{)}{;}{a}{}{,}
\newcolumntype{.}{D{.}{.}{-1}}
\newcolumntype{d}[1]{D{.}{.}{#1}}
\htmladdtonavigation{
  \htmladdnormallink{%
    \htmladdimg{http://gking.harvard.edu/pics/home.gif}}
  {http://gking.harvard.edu/}}
\newcommand{\MatchIt}{{\sc MatchIt}}
\newcommand{\hlink}{\htmladdnormallink}
\newcommand{\Sref}[1]{Section~\ref{#1}}
\newcommand{\fullrvers}{2.5.1}
\newcommand{\rvers}{2.5}
\newcommand{\rwvers}{R-2.5.1}
%\renewcommand{\bibentry}{\citealt}

\bodytext{ BACKGROUND="http://gking.harvard.edu/pics/temple.jpg"}
\setcounter{tocdepth}{2}
 \begin{document}\section{{\tt dims}: Return Dimensions of Vectors, Arrays, and Data Frames}\label{ss:dims}
\keyword{file}{dims}
\begin{Description}\relax
Retrieve the dimensions of a vector, array, or data frame.
\end{Description}
\begin{Usage}
\begin{verbatim}
dims(x)
\end{verbatim}
\end{Usage}
\begin{Arguments}
\begin{ldescription}
\item[\code{x}] An R object.  For example, a vector, matrix, array, or data 
frame.
\end{ldescription}
\end{Arguments}
\begin{Value}
The function \code{dims} performs exactly the same as \code{dim}, and 
additionally returns the \code{length} of vectors (treating them as 
one-dimensional arrays).
\end{Value}
\begin{Author}\relax
Olivia Lau <\email{olau@fas.harvard.edu}>
\end{Author}
\begin{SeeAlso}\relax
\code{dim}, \code{length}
\end{SeeAlso}
\begin{Examples}
\begin{ExampleCode}
a <- 1:12
dims(a)

a <- matrix(1, nrow = 4, ncol = 9)
dims(a)
\end{ExampleCode}
\end{Examples}

\end{document}
