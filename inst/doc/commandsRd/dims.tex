\include{zinput} \begin{document}\section{{\tt dims}: Return Dimensions of Vectors, Arrays, and Data Frames}\label{ss:dims}
\keyword{file}{dims}
\begin{Description}\relax
Retrieve the dimensions of a vector, array, or data frame.
\end{Description}
\begin{Usage}
\begin{verbatim}
dims(x)
\end{verbatim}
\end{Usage}
\begin{Arguments}
\begin{ldescription}
\item[\code{x}] An R object.  For example, a vector, matrix, array, or data 
frame.
\end{ldescription}
\end{Arguments}
\begin{Value}
The function \code{dims} performs exactly the same as \code{dim}, and 
additionally returns the \code{length} of vectors (treating them as 
one-dimensional arrays).
\end{Value}
\begin{Author}\relax
Olivia Lau <\email{olau@fas.harvard.edu}>
\end{Author}
\begin{SeeAlso}\relax
\code{dim}, \code{length}
\end{SeeAlso}
\begin{Examples}
\begin{ExampleCode}
a <- 1:12
dims(a)

a <- matrix(1, nrow = 4, ncol = 9)
dims(a)
\end{ExampleCode}
\end{Examples}

\end{document}
