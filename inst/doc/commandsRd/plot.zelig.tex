\documentclass[oneside,letterpaper,12pt]{book}
\usepackage{Rd}
\usepackage{/usr/lib64/R/share/texmf/Sweave}
\usepackage{bibentry}
\usepackage{upquote}
\usepackage{graphicx}
\usepackage{natbib}
\usepackage[reqno]{amsmath}
\usepackage{amssymb}
\usepackage{amsfonts}
\usepackage{amsmath}
\usepackage{verbatim}
\usepackage{epsf}
\usepackage{url}
\usepackage{html}
\usepackage{dcolumn}
\usepackage{multirow}
\usepackage{fullpage}
\usepackage{lscape}
\usepackage[all]{xy}
% \usepackage[pdftex, bookmarksopen=true,bookmarksnumbered=true,
%   linkcolor=webred]{hyperref}
\bibpunct{(}{)}{;}{a}{}{,}
\newcolumntype{.}{D{.}{.}{-1}}
\newcolumntype{d}[1]{D{.}{.}{#1}}
\htmladdtonavigation{
  \htmladdnormallink{%
    \htmladdimg{http://gking.harvard.edu/pics/home.gif}}
  {http://gking.harvard.edu/}}
\newcommand{\MatchIt}{{\sc MatchIt}}
\newcommand{\hlink}{\htmladdnormallink}
\newcommand{\Sref}[1]{Section~\ref{#1}}
\newcommand{\fullrvers}{2.5.1}
\newcommand{\rvers}{2.5}
\newcommand{\rwvers}{R-2.5.1}
%\renewcommand{\bibentry}{\citealt}

\bodytext{ BACKGROUND="http://gking.harvard.edu/pics/temple.jpg"}
\setcounter{tocdepth}{2}
 \begin{document}\section{{\tt plot.zelig}: Graphing Quantities of Interest}\label{ss:plot.zelig}
\aliasA{plot}{plot.zelig}{plot}
\keyword{hplot}{plot.zelig}
\begin{Description}\relax
The \code{zelig} method for the generic \code{plot}
command generates default plots for \code{\LinkA{sim}{sim}} output with
one-observation values in \code{x} and \code{x1}.
\end{Description}
\begin{Usage}
\begin{verbatim}
## S3 method for class 'zelig':
plot(x, xlab = "", user.par = FALSE, ...)
\end{verbatim}
\end{Usage}
\begin{Arguments}
\begin{ldescription}
\item[\code{x}] stored output from \code{\LinkA{sim}{sim}}.  If the \code{x\$x} or
\code{x\$x1} values stored in the object contain more than one
observation, \code{plot.zelig} will return an error.  For linear or
generalized linear models with more than one observation in \code{x\$x}
and optionally \code{x\$x1}, you may use \code{\LinkA{plot.ci}{plot.ci}}.  
\item[\code{xlab}] a character string for the x-axis label for all graphs.
\item[\code{user.par}] a logical value indicating whether to use the default
Zelig plotting parameters (\code{user.par = FALSE}) or
user-defined parameters (\code{user.par = TRUE}), set using the
\code{par} function prior to plotting. 
\item[\code{...}] Additional parameters passed to \code{plot.default}.
Because \code{plot.zelig} primarily produces diagnostic plots, many
of these parameters are hard-coded for convenience and
presentation. 
\end{ldescription}
\end{Arguments}
\begin{Value}
Depending on the class of model selected, \code{plot.zelig} will
return an on-screen window with graphs of the various quantities of
interest.  You may save these plots using the commands described in
the Zelig manual (available at \url{http://gking.harvard.edu/zelig}).
\end{Value}
\begin{Author}\relax
Kosuke Imai <\email{kimai@princeton.edu}>; Gary King
<\email{king@harvard.edu}>; Olivia Lau <\email{olau@fas.harvard.edu}>
\end{Author}
\begin{SeeAlso}\relax
The full Zelig manual at
\url{http://gking.harvard.edu/zelig} and \code{plot}, \code{lines},
and \code{par}.
\end{SeeAlso}

\end{document}
