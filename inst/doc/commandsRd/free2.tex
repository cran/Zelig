\documentclass[oneside,letterpaper,12pt]{book}
\usepackage{Rd}
\usepackage{/usr/lib64/R/share/texmf/Sweave}
\usepackage{bibentry}
\usepackage{upquote}
\usepackage{graphicx}
\usepackage{natbib}
\usepackage[reqno]{amsmath}
\usepackage{amssymb}
\usepackage{amsfonts}
\usepackage{amsmath}
\usepackage{verbatim}
\usepackage{epsf}
\usepackage{url}
\usepackage{html}
\usepackage{dcolumn}
\usepackage{multirow}
\usepackage{fullpage}
\usepackage{lscape}
\usepackage[all]{xy}
% \usepackage[pdftex, bookmarksopen=true,bookmarksnumbered=true,
%   linkcolor=webred]{hyperref}
\bibpunct{(}{)}{;}{a}{}{,}
\newcolumntype{.}{D{.}{.}{-1}}
\newcolumntype{d}[1]{D{.}{.}{#1}}
\htmladdtonavigation{
  \htmladdnormallink{%
    \htmladdimg{http://gking.harvard.edu/pics/home.gif}}
  {http://gking.harvard.edu/}}
\newcommand{\MatchIt}{{\sc MatchIt}}
\newcommand{\hlink}{\htmladdnormallink}
\newcommand{\Sref}[1]{Section~\ref{#1}}
\newcommand{\fullrvers}{2.5.1}
\newcommand{\rvers}{2.5}
\newcommand{\rwvers}{R-2.5.1}
%\renewcommand{\bibentry}{\citealt}

\bodytext{ BACKGROUND="http://gking.harvard.edu/pics/temple.jpg"}
\setcounter{tocdepth}{2}
 \begin{document}\section{{\tt free2}: Freedom of Speech Data}\label{ss:free2}
\keyword{datasets}{free2}
\begin{Description}\relax
Selection of individual-level survey data for freedom of speech.
\end{Description}
\begin{Usage}
\begin{verbatim}data(free2)\end{verbatim}
\end{Usage}
\begin{Details}\relax
A table with 150 observations and 12 variables. \Itemize{
\item[sex] 1 for men and 0 for women
\item[age] Age of respondent in years
\item[educ] Levels of education, coded as a numeric variable with
\Itemize{
\item[1] No formal education
\item[2] Less than primary school education
\item[3] Completed primary school
\item[4] Completed secondary school
\item[5] Completed high school
\item[6] Completed college
\item[7] Completed post-graduate degree
}

\item[country] Character strings consisting of "Oceana",
"Eurasia", and "Eastasia", after Orwell's \emph{1984}.
\item[y] Self assessment (see below).
\item[v1-v6] Response to vignettes (see below).
}
Survey respondents were asked in almost the same language for a
self-assessment and for an assessment of several hypothetical persons
described by written vignettes.  The self assessment (\code{self}, in
the data set), "How free do you think [name/you] [is/are] to express
[him-her/your]self without fear of government reprisal?" was first
asked of the survey respondent with respect to him or herself, and
then after each of vignette.  The possible response categories are:  \Itemize{
\item[1] Completely free
\item[2] Very free
\item[3] Moderately free
\item[4] Slightly free
\item[5] Not free at all
}
The vignettes, ordered from most free to least free, are:
\Itemize{
\item[vign1] [Kay] does not like many of the government's
policies. She frequently publishes her opinion in newspapers,
criticizing decisions by officials and calling for change. She sees
little reason these actions could lead to government reprisal.

\item[vign2] [Michael] disagrees with many of the government's
policies. Though he knows criticism is frowned upon, he doesn't
believe the government would punish someone for expressing critical
views. He makes his opinion known on most issues without regard to
who is listening.

\item[vign3] [Bob] has political views at odds with the
government. He has heard of people occasionally being arrested for
speaking out against the government, and government leaders
sometimes make political speeches condemning those who criticize. He
sometimes writes letters to newspapers about politics, but he is
careful not to use his real name.

\item[vign4] [Connie] does not like the government's stance on many
issues. She has a friend who was arrested for being too openly
critical of governmental leaders, and so she avoids voicing her
opinions in public places.

\item[vign5] [Vito] disagrees with many of the government's
policies, and is very careful about whom he says this to, reserving
his real opinions for family and close friends only. He knows
several men who have been taken away by government officials for
saying negative things in public.

\item[vign6] [Sonny] lives in fear of being harassed for his
political views. Everyone he knows who has spoken out against the
government has been arrested or taken away. He never says a word
about anything the government does, not even when he is at home
alone with his family. 
}
\end{Details}
\begin{References}\relax
\emph{WHO's World Health Survey}
by Lydia Bendib, Somnath Chatterji, Alena Petrakova, Ritu Sadana,
Joshua A. Salomon, Margie Schneider, Bedirhan Ustun, Maria
Villanueva

Jonathan Wand, Gary King and Olivia Lau. (2007) ``Anchors: Software for
Anchoring Vignettes''. \emph{Journal of Statistical Software}.  Forthcoming.
copy at http://wand.stanford.edu/research/anchors-jss.pdf

Gary King and Jonathan Wand.  "Comparing Incomparable Survey
Responses: New Tools for Anchoring Vignettes," Political Analysis, 15,
1 (Winter, 2007): Pp. 46-66,
copy at http://gking.harvard.edu/files/abs/c-abs.shtml.
\end{References}

\end{document}
