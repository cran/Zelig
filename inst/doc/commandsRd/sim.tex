\documentclass[oneside,letterpaper,12pt]{book}
\usepackage{Rd}
\usepackage{/usr/lib64/R/share/texmf/Sweave}
\usepackage{bibentry}
\usepackage{upquote}
\usepackage{graphicx}
\usepackage{natbib}
\usepackage[reqno]{amsmath}
\usepackage{amssymb}
\usepackage{amsfonts}
\usepackage{amsmath}
\usepackage{verbatim}
\usepackage{epsf}
\usepackage{url}
\usepackage{html}
\usepackage{dcolumn}
\usepackage{multirow}
\usepackage{fullpage}
\usepackage{lscape}
\usepackage[all]{xy}
% \usepackage[pdftex, bookmarksopen=true,bookmarksnumbered=true,
%   linkcolor=webred]{hyperref}
\bibpunct{(}{)}{;}{a}{}{,}
\newcolumntype{.}{D{.}{.}{-1}}
\newcolumntype{d}[1]{D{.}{.}{#1}}
\htmladdtonavigation{
  \htmladdnormallink{%
    \htmladdimg{http://gking.harvard.edu/pics/home.gif}}
  {http://gking.harvard.edu/}}
\newcommand{\MatchIt}{{\sc MatchIt}}
\newcommand{\hlink}{\htmladdnormallink}
\newcommand{\Sref}[1]{Section~\ref{#1}}
\newcommand{\fullrvers}{2.5.1}
\newcommand{\rvers}{2.5}
\newcommand{\rwvers}{R-2.5.1}
%\renewcommand{\bibentry}{\citealt}

\bodytext{ BACKGROUND="http://gking.harvard.edu/pics/temple.jpg"}
\setcounter{tocdepth}{2}
 \begin{document}\section{{\tt sim}: Simulating Quantities of Interest}\label{ss:sim}
\keyword{file}{sim}
\begin{Description}\relax
Simulate quantities of interest from the estimated model
output from \code{zelig()} given specified values of explanatory
variables established in \code{setx()}.  For classical \emph{maximum
likelihood} models, \code{sim()} uses asymptotic normal
approximation to the log-likelihood.  For \emph{Bayesian models},
Zelig simulates quantities of interest from the posterior density,
whenever possible.  For \emph{robust Bayesian models}, simulations
are drawn from the identified class of Bayesian posteriors.
Alternatively, you may generate quantities of interest using
bootstrapped parameters.
\end{Description}
\begin{Usage}
\begin{verbatim}
s.out <- sim(object, x, x1 = NULL, num = c(1000, 100), prev = NULL, 
             bootstrap = FALSE,  bootfn = NULL, ...)
\end{verbatim}
\end{Usage}
\begin{Arguments}
\begin{ldescription}
\item[\code{object}] the output object from \code{\LinkA{zelig}{zelig}}. 
\item[\code{x}] values of explanatory variables used for simulation,
generated by \code{\LinkA{setx}{setx}}.  
\item[\code{x1}] optional values of explanatory variables (generated by a
second call of \code{\LinkA{setx}{setx}}), used to simulate first
differences and risk ratios.  (Not available for conditional
prediction.) 
\item[\code{num}] the number of simulations, i.e., posterior draws.  If the
\code{num} argument is omitted, \code{sim} draws 1,000
simulations by if \code{bootstrap = FALSE} (the default), or 100
simulations if \code{bootstrap = TRUE}.  You may increase this
value to improve accuracy.  (Not available for conditional
prediction.) 
\item[\code{bootstrap}] a logical value indicating if parameters
should be generated by re-fitting the model for bootstrapped
data, rather than from the likelihood or posterior.  (Not
available for conditional prediction.) 
\item[\code{bootfn}] a function which governs how the data is
sampled, re-fits the model, and returns the bootstrapped model
parameters.  If \code{bootstrap = TRUE} and \code{bootfn = NULL},
\code{\LinkA{sim}{sim}} will sample observations from the original data
(with
replacement) until it creates a sampled dataset with the same
number of observations as the original data.  Alternative
bootstrap methods include sampling the residuals rather than the
observations, weighted sampling, and parametric bootstrapping.
(Not available for conditional prediction.) 
\item[\code{...}] additional optional arguments passed to
\code{boot}. 
\end{ldescription}
\end{Arguments}
\begin{Value}
The output stored in \code{s.out} varies by model.  Use the
\code{names} command to view the output stored in \code{s.out}.
Common elements include:  normal-bracket109bracket-normal
\begin{ldescription}
\item[\code{x}] the \code{\LinkA{setx}{setx}} values for the explanatory variables,
used to calculate the quantities of interest (expected values,
predicted values, etc.). 
\item[\code{x1}] the optional \code{\LinkA{setx}{setx}} object used to simulate
first differences, and other model-specific quantities of
interest, such as risk-ratios.
\item[\code{call}] the options selected for \code{\LinkA{sim}{sim}}, used to
replicate quantities of interest. 
\item[\code{zelig.call}] the original command and options for
\code{\LinkA{zelig}{zelig}}, used to replicate analyses. 
\item[\code{num}] the number of simulations requested. 
\item[\code{par}] the parameters (coefficients, and additional
model-specific parameters).  You may wish to use the same set of
simulated parameters to calculate quantities of interest rather
than simulating another set.
\item[\code{qi\$ev}] simulations of the expected values given the
model and \code{x}. 
\item[\code{qi\$pr}] simulations of the predicted values given by the
fitted values. 
\item[\code{qi\$fd}] simulations of the first differences (or risk
difference for binary models) for the given \code{x} and \code{x1}.
The difference is calculated by subtracting the expected values
given \code{x} from the expected values given \code{x1}.  (If do not
specify \code{x1}, you will not get first differences or risk
ratios.) 
\item[\code{qi\$rr}] simulations of the risk ratios for binary and
multinomial models.  See specific models for details.
\item[\code{qi\$ate.ev}] simulations of the average expected
treatment effect for the treatment group, using conditional
prediction. Let \eqn{t_i}{} be a binary explanatory variable defining
the treatment (\eqn{t_i=1}{}) and control (\eqn{t_i=0}{}) groups.  Then the
average expected treatment effect for the treatment group is
\deqn{ \frac{1}{n}\sum_{i=1}^n [ \, Y_i(t_i=1) -
E[Y_i(t_i=0)] \mid t_i=1 \,],}{} 
where \eqn{Y_i(t_i=1)}{} is the value of the dependent variable for
observation \eqn{i}{} in the treatment group.  Variation in the
simulations are due to uncertainty in simulating \eqn{E[Y_i(t_i=0)]}{},
the counterfactual expected value of \eqn{Y_i}{} for observations in the
treatment group, under the assumption that everything stays the
same except that the treatment indicator is switched to \eqn{t_i=0}{}. 
\item[\code{qi\$ate.pr}] simulations of the average predicted
treatment effect for the treatment group, using conditional
prediction. Let \eqn{t_i}{} be a binary explanatory variable defining
the treatment (\eqn{t_i=1}{}) and control (\eqn{t_i=0}{}) groups.  Then the
average predicted treatment effect for the treatment group is
\deqn{ \frac{1}{n}\sum_{i=1}^n [ \, Y_i(t_i=1) -
\widehat{Y_i(t_i=0)} \mid t_i=1 \,],}{} 
where \eqn{Y_i(t_i=1)}{} is the value of the dependent variable for
observation \eqn{i}{} in the treatment group.  Variation in the
simulations are due to uncertainty in simulating
\eqn{\widehat{Y_i(t_i=0)}}{}, the counterfactual predicted value of
\eqn{Y_i}{} for observations in the treatment group, under the
assumption that everything stays the same except that the
treatment indicator is switched to \eqn{t_i=0}{}. 
\end{ldescription}

normal-bracket109bracket-normal

In the case of censored $Y$ in the exponential, Weibull, and lognormal
models, \code{sim} first imputes the uncensored values for $Y$ before
calculating the ATE.  

You may use the \code{\$} operator to extract any of the
above from \code{s.out}.  For example, \code{s.out\$qi\$ev} extracts the
simulated expected values.
\end{Value}
\begin{Author}\relax
Kosuke Imai <\email{kimai@princeton.edu}>; Gary King
<\email{king@harvard.edu}>; Olivia Lau <\email{olau@fas.harvard.edu}>
\end{Author}
\begin{SeeAlso}\relax
The full Zelig at \url{http://gking.harvard.edu/zelig}, and \code{boot}.
\end{SeeAlso}

\end{document}
