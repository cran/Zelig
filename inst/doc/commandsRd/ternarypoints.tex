\documentclass[oneside,letterpaper,12pt]{book}
\usepackage{Rd}
\usepackage{/usr/lib64/R/share/texmf/Sweave}
\usepackage{bibentry}
\usepackage{upquote}
\usepackage{graphicx}
\usepackage{natbib}
\usepackage[reqno]{amsmath}
\usepackage{amssymb}
\usepackage{amsfonts}
\usepackage{amsmath}
\usepackage{verbatim}
\usepackage{epsf}
\usepackage{url}
\usepackage{html}
\usepackage{dcolumn}
\usepackage{multirow}
\usepackage{fullpage}
\usepackage{lscape}
\usepackage[all]{xy}
% \usepackage[pdftex, bookmarksopen=true,bookmarksnumbered=true,
%   linkcolor=webred]{hyperref}
\bibpunct{(}{)}{;}{a}{}{,}
\newcolumntype{.}{D{.}{.}{-1}}
\newcolumntype{d}[1]{D{.}{.}{#1}}
\htmladdtonavigation{
  \htmladdnormallink{%
    \htmladdimg{http://gking.harvard.edu/pics/home.gif}}
  {http://gking.harvard.edu/}}
\newcommand{\MatchIt}{{\sc MatchIt}}
\newcommand{\hlink}{\htmladdnormallink}
\newcommand{\Sref}[1]{Section~\ref{#1}}
\newcommand{\fullrvers}{2.5.1}
\newcommand{\rvers}{2.5}
\newcommand{\rwvers}{R-2.5.1}
%\renewcommand{\bibentry}{\citealt}

\bodytext{ BACKGROUND="http://gking.harvard.edu/pics/temple.jpg"}
\setcounter{tocdepth}{2}
 \begin{document}\section{{\tt ternarypoints}: Adding Points to Ternary Diagrams}\label{ss:ternarypoints}
\keyword{aplot}{ternarypoints}
\begin{Description}\relax
Use \code{ternarypoints} to add points to a ternary diagram generated
using the \code{ternaryplot} function in the vcd library.  Use
ternary diagrams to plot expected values for multinomial choice models
with three categories in the dependent variable.
\end{Description}
\begin{Usage}
\begin{verbatim}
ternarypoints(object, pch = 19, col = "blue", ...)
\end{verbatim}
\end{Usage}
\begin{Arguments}
\begin{ldescription}
\item[\code{object}] The input object must be a matrix with three
columns. 
\item[\code{pch}] The selected type of point.  By default, \code{pch =
    19}, solid disks. 
\item[\code{col}] The color of the points.  By default, \code{col =
    "blue"}. 
\item[\code{...}] Additional parameters passed to \code{points}. 
\end{ldescription}
\end{Arguments}
\begin{Value}
The \code{ternarypoints} command adds points to a previously existing
ternary diagram.  Use \code{ternaryplot} in the \code{vcd} library to
generate the main ternary diagram.
\end{Value}
\begin{Author}\relax
Kosuke Imai <\email{kimai@princeton.edu}>; Gary King
<\email{king@harvard.edu}>; Olivia Lau <\email{olau@fas.harvard.edu}>
\end{Author}
\begin{SeeAlso}\relax
The full Zelig manual at
\url{http://gking.harvard.edu/zelig}, \code{points}, and
\code{\LinkA{ternaryplot}{ternaryplot}}.
\end{SeeAlso}

\end{document}
