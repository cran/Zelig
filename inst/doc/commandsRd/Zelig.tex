\documentclass[oneside,letterpaper,12pt]{book}
\usepackage{Rd}
\usepackage{/usr/lib64/R/share/texmf/Sweave}
\usepackage{bibentry}
\usepackage{upquote}
\usepackage{graphicx}
\usepackage{natbib}
\usepackage[reqno]{amsmath}
\usepackage{amssymb}
\usepackage{amsfonts}
\usepackage{amsmath}
\usepackage{verbatim}
\usepackage{epsf}
\usepackage{url}
\usepackage{html}
\usepackage{dcolumn}
\usepackage{multirow}
\usepackage{fullpage}
\usepackage{lscape}
\usepackage[all]{xy}
% \usepackage[pdftex, bookmarksopen=true,bookmarksnumbered=true,
%   linkcolor=webred]{hyperref}
\bibpunct{(}{)}{;}{a}{}{,}
\newcolumntype{.}{D{.}{.}{-1}}
\newcolumntype{d}[1]{D{.}{.}{#1}}
\htmladdtonavigation{
  \htmladdnormallink{%
    \htmladdimg{http://gking.harvard.edu/pics/home.gif}}
  {http://gking.harvard.edu/}}
\newcommand{\MatchIt}{{\sc MatchIt}}
\newcommand{\hlink}{\htmladdnormallink}
\newcommand{\Sref}[1]{Section~\ref{#1}}
\newcommand{\fullrvers}{2.5.1}
\newcommand{\rvers}{2.5}
\newcommand{\rwvers}{R-2.5.1}
%\renewcommand{\bibentry}{\citealt}

\bodytext{ BACKGROUND="http://gking.harvard.edu/pics/temple.jpg"}
\setcounter{tocdepth}{2}
 \begin{document}\section{{\tt zelig}: Estimating a Statistical Model}\label{ss:zelig}
\keyword{file}{zelig}
\begin{Description}\relax
The \code{zelig} command estimates a variety of statistical
models.  Use \code{zelig} output with \code{setx} and \code{sim} to compute
quantities of interest, such as predicted probabilities, expected values, and
first differences, along with the associated measures of uncertainty
(standard errors and confidence intervals).
\end{Description}
\begin{Usage}
\begin{verbatim}
z.out <- zelig(formula, model, data, by, save.data, cite, ...) 
\end{verbatim}
\end{Usage}
\begin{Arguments}
\begin{ldescription}
\item[\code{formula}] a symbolic representation of the model to be
estimated, in the form \code{y \textasciitilde{}\bsl{}, x1 + x2}, where \code{y} is the
dependent variable and \code{x1} and \code{x2} are the explanatory
variables, and \code{y}, \code{x1}, and \code{x2} are contained in the
same dataset.  (You may include more than two explanatory variables,
of course.)  The \code{+} symbol means ``inclusion'' not
``addition.''  You may also include interaction terms and main
effects in the form \code{x1*x2} without computing them in prior
steps; \code{I(x1*x2)} to include only the interaction term and
exclude the main effects; and quadratic terms in the form
\code{I(x1\textasciicircum{}2)}.  
\item[\code{model}] the name of a statistical model, enclosed in \code{""}.
Type \code{help.zelig("models")} to see a list of currently supported
models.  
\item[\code{data}] the name of a data frame containing the variables
referenced in the formula, or a list of multiply imputed data frames
each having the same variable names and row numbers (created by
\code{mi}). 
\item[\code{save.data}] If is set to "TRUE", the input dataframe will be saved
as an attribute ("zelig.data") of the zelig output object. 
\item[\code{cite}] If is set to "TRUE" (default), the model citation will be
will be printed out when this function is called. 
\item[\code{by}] a factor variable contained in \code{data}.  Zelig will subset
the data frame based on the levels in the \code{by} variable, and
estimate a model for each subset.  This a particularly powerful option
which will allow you to save a considerable amount of effort.  For
example, to run the same model on all fifty states, you could type:
\code{z.out <- zelig(y \textasciitilde{} x1 + x2, data = mydata, model = "ls", by = "state")}
You may also use \code{by} to run models using MatchIt subclass.  
\item[\code{...}] additional arguments passed to \code{zelig},
depending on the model to be estimated. 
\end{ldescription}
\end{Arguments}
\begin{Value}
Depending on the class of model selected, \code{zelig} will return
an object with elements including \code{coefficients}, \code{residuals},
and \code{formula} which may be summarized using
\code{summary(z.out)} or individually extracted using, for example,
\code{z.out\$coefficients}.  See the specific models listed above
for additional output values, or simply type \code{names(z.out)}.
\end{Value}
\begin{Author}\relax
Kosuke Imai <\email{kimai@princeton.edu}>; Gary King
<\email{king@harvard.edu}>; Olivia Lau <\email{olau@fas.harvard.edu}>
\end{Author}
\begin{SeeAlso}\relax
The full Zelig manual is available at
\url{http://gking.harvard.edu/zelig}.
\end{SeeAlso}

\end{document}
