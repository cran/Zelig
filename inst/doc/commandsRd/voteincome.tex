\documentclass[oneside,letterpaper,12pt]{book}
\usepackage{Rd}
\usepackage{/usr/lib64/R/share/texmf/Sweave}
\usepackage{bibentry}
\usepackage{upquote}
\usepackage{graphicx}
\usepackage{natbib}
\usepackage[reqno]{amsmath}
\usepackage{amssymb}
\usepackage{amsfonts}
\usepackage{amsmath}
\usepackage{verbatim}
\usepackage{epsf}
\usepackage{url}
\usepackage{html}
\usepackage{dcolumn}
\usepackage{multirow}
\usepackage{fullpage}
\usepackage{lscape}
\usepackage[all]{xy}
% \usepackage[pdftex, bookmarksopen=true,bookmarksnumbered=true,
%   linkcolor=webred]{hyperref}
\bibpunct{(}{)}{;}{a}{}{,}
\newcolumntype{.}{D{.}{.}{-1}}
\newcolumntype{d}[1]{D{.}{.}{#1}}
\htmladdtonavigation{
  \htmladdnormallink{%
    \htmladdimg{http://gking.harvard.edu/pics/home.gif}}
  {http://gking.harvard.edu/}}
\newcommand{\MatchIt}{{\sc MatchIt}}
\newcommand{\hlink}{\htmladdnormallink}
\newcommand{\Sref}[1]{Section~\ref{#1}}
\newcommand{\fullrvers}{2.5.1}
\newcommand{\rvers}{2.5}
\newcommand{\rwvers}{R-2.5.1}
%\renewcommand{\bibentry}{\citealt}

\bodytext{ BACKGROUND="http://gking.harvard.edu/pics/temple.jpg"}
\setcounter{tocdepth}{2}
 \begin{document}\section{{\tt voteincome}: Sample Turnout and Demographic Data from the 2000 Current Population Survey}\label{ss:voteincome}
\keyword{datasets}{voteincome}
\begin{Description}\relax
This data set contains turnout and demographic data from a sample of respondents to the 2000 Current Population Survey (CPS). The states represented are South Carolina and Arkansas. The data represent only a sample and results from this example should not be used in publication.
\end{Description}
\begin{Usage}
\begin{verbatim}data(voteincome)\end{verbatim}
\end{Usage}
\begin{Format}\relax
A data frame containing 7 variables ("state", "year", "vote", "income", "education", "age", "female") and 1500 observations.  
\describe{
\item[\code{state}] a factor variable with levels equal to "AR" (Arkansas) and "SC" (South Carolina)
\item[\code{year}] an integer vector
\item[\code{vote}] an integer vector taking on values "1" (Voted) and "0" (Did Not Vote)
\item[\code{income}] an integer vector ranging from "4" (Less than \$5000) to "17" (Greater than \$75000) denoting family income. See the CPS codebook for more information on variable coding
\item[\code{education}] an  integer vector ranging from "1" (Less than High School Education) to "4" (More than a College Education). See the CPS codebook for more information on variable coding
\item[\code{age}] an integer vector ranging from "18" to "85"
\item[\code{female}] an integer vector taking on values "1" (Female) and "0" (Male)
}
\end{Format}
\begin{Source}\relax
Census Bureau Current Population Survey
\end{Source}
\begin{References}\relax
\url{http://www.census.gov/cps}
\end{References}

\end{document}
