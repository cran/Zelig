\documentclass[oneside,letterpaper,12pt]{book}
\usepackage{Rd}
\usepackage{/usr/lib64/R/share/texmf/Sweave}
\usepackage{bibentry}
\usepackage{upquote}
\usepackage{graphicx}
\usepackage{natbib}
\usepackage[reqno]{amsmath}
\usepackage{amssymb}
\usepackage{amsfonts}
\usepackage{amsmath}
\usepackage{verbatim}
\usepackage{epsf}
\usepackage{url}
\usepackage{html}
\usepackage{dcolumn}
\usepackage{multirow}
\usepackage{fullpage}
\usepackage{lscape}
\usepackage[all]{xy}
% \usepackage[pdftex, bookmarksopen=true,bookmarksnumbered=true,
%   linkcolor=webred]{hyperref}
\bibpunct{(}{)}{;}{a}{}{,}
\newcolumntype{.}{D{.}{.}{-1}}
\newcolumntype{d}[1]{D{.}{.}{#1}}
\htmladdtonavigation{
  \htmladdnormallink{%
    \htmladdimg{http://gking.harvard.edu/pics/home.gif}}
  {http://gking.harvard.edu/}}
\newcommand{\MatchIt}{{\sc MatchIt}}
\newcommand{\hlink}{\htmladdnormallink}
\newcommand{\Sref}[1]{Section~\ref{#1}}
\newcommand{\fullrvers}{2.5.1}
\newcommand{\rvers}{2.5}
\newcommand{\rwvers}{R-2.5.1}
%\renewcommand{\bibentry}{\citealt}

\bodytext{ BACKGROUND="http://gking.harvard.edu/pics/temple.jpg"}
\setcounter{tocdepth}{2}
 \begin{document}\section{{\tt parse.par}: Select and reshape parameter vectors}\label{ss:parse.par}
\keyword{utilities}{parse.par}
\begin{Description}\relax
The \code{parse.par} function reshapes parameter vectors for
comfortability with the output matrix from \code{\LinkA{model.matrix.multiple}{model.matrix.multiple}}. 
Use \code{parse.par} to identify sets of parameters; for example, within
optimization functions that require vector input, or within \code{qi}
functions that take matrix input of all parameters as a lump.
\end{Description}
\begin{Usage}
\begin{verbatim}
parse.par(par, terms, shape = "matrix", eqn = NULL)
\end{verbatim}
\end{Usage}
\begin{Arguments}
\begin{ldescription}
\item[\code{par}] the vector (or matrix) of parameters
\item[\code{terms}] the terms from either \code{\LinkA{model.frame.multiple}{model.frame.multiple}} or 
\code{\LinkA{model.matrix.multiple}{model.matrix.multiple}}
\item[\code{shape}] a character string (either \code{"matrix"} or \code{"vector"})
that identifies the type of output structure
\item[\code{eqn}] a character string (or strings) that identify the
parameters that you would like to subset from the larger \code{par}
structure
\end{ldescription}
\end{Arguments}
\begin{Value}
A matrix or vector of the sub-setted (and reshaped) parameters for the specified
parameters given in \code{"eqn"}.   By default, \code{eqn = NULL}, such that all systematic
components are selected.  (Systematic components have \code{ExpVar = TRUE} in the appropriate 
\code{describe.model} function.)  

If an ancillary parameter (for which \code{ExpVar = FALSE} in
\code{describe.model}) is specified in \code{eqn}, it is
always returned as a vector (ignoring \code{shape}).  (Ancillary
parameters are all parameters that have intercept only formulas.)
\end{Value}
\begin{Author}\relax
Kosuke Imai <\email{kimai@princeton.edu}>; Gary King
<\email{king@harvard.edu}>; Olivia Lau <\email{olau@fas.harvard.edu}>; Ferdinand Alimadhi
<\email{falimadhi@iq.harvard.edu}>
\end{Author}
\begin{SeeAlso}\relax
\code{\LinkA{model.matrix.multiple}{model.matrix.multiple}}, \code{\LinkA{parse.formula}{parse.formula}} and the full Zelig manual at
\url{http://gking.harvard.edu/zelig}
\end{SeeAlso}
\begin{Examples}
\begin{ExampleCode}
# Let's say that the name of the model is "bivariate.probit", and
# the corresponding describe function is describe.bivariate.probit(), 
# which identifies mu1 and mu2 as systematic components, and an 
# ancillary parameter rho, which may be parameterized, but is estimated 
# as a scalar by default.  Let par be the parameter vector (including 
# parameters for rho), formulae a user-specified formula, and mydata
# the user specified data frame.  

# Acceptable combinations of parse.par() and model.matrix() are as follows:
## Setting up
## Not run: 
data(sanction)
formulae <- cbind(import, export) ~ coop + cost + target
fml <- parse.formula(formulae, model = "bivariate.probit")
D <- model.frame(fml, data = sanction)
terms <- attr(D, "terms")

## Intuitive option
Beta <- parse.par(par, terms, shape = "vector", eqn = c("mu1", "mu2"))
X <- model.matrix(fml, data = D, shape = "stacked", eqn = c("mu1", "mu2")
eta <- X 

## Memory-efficient (compact) option (default)
Beta <- parse.par(par, terms, eqn = c("mu1", "mu2"))    
X <- model.matrix(fml, data = D, eqn = c("mu1", "mu2"))
eta <- X 

## Computationally-efficient (array) option
Beta <- parse.par(par, terms, shape = "vector", eqn = c("mu1", "mu2"))
X <- model.matrix(fml, data = D, shape = "array", eqn = c("mu1", "mu2"))
eta <- apply(X, 3, '
## End(Not run)\end{ExampleCode}
\end{Examples}

\end{document}
