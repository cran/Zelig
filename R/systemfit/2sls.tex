\documentclass[12pt]{book}%
\usepackage{amsfonts}
\usepackage{amsmath}
\usepackage{amssymb}
\usepackage{graphicx}
\usepackage{fullpage}
\usepackage{multirow}
\usepackage{hyperref}%
\usepackage{bibentry}
\begin{document}
\subsection{\texttt{2sls}: Two Stage Least Squares}
\label{2sls}
\texttt{2sls} provides consistent estimates for linear regression models with 
some explanatory variable (the instrumental variable) 
correlated with the error term. 
In this situation, ordinary least squares fails to provide consistent 
estimates. The name two-stage least squares stems from the two regressions 
in the estimation procedure. In stage one, an ordinary least squares 
prediction of the instrumental variable is obtained from regressing it on
the instrument variables. In stage two, the coefficients of interst are 
estimated using ordinary least square after substituting the instrumental 
variable by its predictions from stage one. 

\subsubsection{Syntax}
\begin{verbatim}
> fml <- list ("mu"  = Y ~ X + Z,
               "inst" = Z ~ W + X)
> z.out <- zelig(formula = fml, model = "2sls", data = mydata)
> x.out <- setx(z.out)
> s.out <- sim(z.out, x = x.out)
\end{verbatim}
\subsubsection{Inputs}
\texttt{2sls} regression take the following inputs:
\begin{itemize}
\item \texttt{formula}:a list of the main equation and instrumental variable 
equation. The first object in the list \texttt{mu} corresponds to the 
regression model needs to be estimated. The second list object \texttt{inst} 
specifies the regression model for the instrumental variable \texttt{Z}.
For example:
\begin{verbatim}
> fml <- list ("mu"  = Y ~ X + Z,
	       "inst" = Z ~ W + X)
\end{verbatim}
\begin{itemize}
\item \texttt{Y}: the dependent variable of interest.
\item \texttt{Z}: the instrumental variable.
\item \texttt{W}: exogenous instrument variables.
\end{itemize}
\end{itemize}
\subsubsection{Additional Inputs}
\texttt{2sls} takes the following additional inputs for model
specifications:
\begin{itemize}
\item \texttt{TX}: an optional matrix to transform the regressor
matrix and, hence, also the coefficient vector (see \ref{details}). Default is \texttt{NULL}.
\item \texttt{rcovformula}: formula to calculate the estimated residual covariance
matrix (see \ref{details}). Default is equal to 1.
\item \texttt{probdfsys}: use the degrees of freedom of the whole system
(in place of the degrees of freedom of the single equation to calculate probability
values for the t-test of individual parameters. 
\item \texttt{single.eq.sigma}: use different $\sigma^2$ for each single
equation to calculate the covariance matrix and the standard errors of the coefficients.
\item \texttt{solvetol}: tolerance level for detecting linear dependencies when 
inverting a matrix or calculating a determinant. Default is \texttt {solvetol}=.Machine\$double.eps.
\item \texttt{saveMemory}: logical. Save memory by omitting some calculation that are
not crucial for the basic estimate (e.g McElroy's $R^2$).
\end{itemize}
\subsubsection{Details}
\label{details}
\begin{itemize}
\item \texttt{TX}: The matrix \texttt{TX} transforms the regressor matrix 
($X$) by $X\ast=X \times TX$. Thus,
the vector of coefficients is now $b=TX \times b\ast$ where $b$ is the 
original(stacked) 
vector of all coefficients and $b\ast$ is the new coefficient vector 
that is estimated instead.
Thus, the elements of vector $b$ and $b_i = \sum_j TX_{ij}\times b_j\ast$. The $TX$ matrix can be
used to change the order of the coefficients and also to restrict coefficients (if $TX$ has 
less columns than it has rows).
\item \texttt{rcovformula}: The formula to calculate the estimated covariance matrix of the residuals($\hat{\Sigma}$)can be one
of the following (see Judge et al., 1955, p.469):
if \texttt{rcovformula}= 0:
\begin{eqnarray*}
\hat{\sigma_{ij}}= \frac{\hat{e_i}\prime\hat{e_j}}{T}
\end{eqnarray*}
if \texttt{rcovformula}= 1 or \texttt{rcovformula}='geomean':
\begin{eqnarray*}
\hat{\sigma_{ij}}= \frac{\hat{e_i}\prime\hat{e_j}}{\sqrt{(T-k_i)\times (T-k_j)}}
\end{eqnarray*}
if \texttt{rcovformula}= 2 or \texttt{rcovformula}='Theil':
\begin{eqnarray*}
\hat{\sigma_{ij}}= \frac{\hat{e_i}\prime\hat{e_j}}{T-k_i-k_j+tr[X_i(X_i\prime X_i)^{-1}X_i\prime X_j(X_j\prime X_j)^{-1}X_j\prime]}
\end{eqnarray*}
if \texttt{rcovformula}= 3 or \texttt{rcovformula}='max':
\begin{eqnarray*}
\hat{\sigma_{ij}}= \frac{\hat{e_i}\prime\hat{e_j}}{T-max(k_i,k_j)}
\end{eqnarray*}
If $i = j$, formula 1, 2, and 3 are equal. All these three formulas yield unbiased estimators
for the diagonal elements of the residual covariance matrix. If $i neq j$, only formula 2
yields an unbiased estimator for the residual covariance matrix, but it is not necessarily
positive semidefinit. Thus, it is doubtful whether formula 2 is really superior to formula 1
\end{itemize}
\subsubsection{Examples}
\subsubsection{Model}
Let's consider the following regression model,
\begin{eqnarray*}
Y_i=X_i\beta + Z_i\gamma + \epsilon_i, \quad  i=1,\ldots,N
\end{eqnarray*}
where $Y_i$ is the dependent variable, 
$X_i = (X_{1i},\ldots, X_{Ni})$ is the vector of explanatory variables, 
$\beta$ is the vector of coefficients of the explanatory variables $X_i$, 
$Z_i$ 
is the problematic explanatory variable, and $\gamma$ is the coefficient
 of $Z_i$.  In the equation, there is a direct dependence of $Z_i$ 
on the structural disturbances of $\epsilon$.
\begin{itemize}
\item The \emph{stochastic component} is given by
\begin{eqnarray*}
\epsilon_i  &  \sim & {\cal N}(0, \sigma^2), \quad {\rm and} \quad
{\rm cov}(Z_i, \epsilon_i) \ne 0,
\end{eqnarray*}
\item The \emph{systematic component} is given by:
\begin{eqnarray*}
\mu_{i}= E(Y_i)= X_{i}\beta + Z_i\gamma,
\end{eqnarray*}
\end{itemize}
\noindent To correct the problem caused by the correlation of $Z_i$ and $\epsilon$, 
two stage least squares utilizes two steps:
\begin{itemize}
\item \emph{Stage 1}: A new instrumental variable $\hat{Z}$ is created 
for $Z_i$ which is the ordinary least squares predictions from regressing 
$Z_i$ on a set of exogenous instruments $W$ and $X$.
\begin{eqnarray*}
\widehat{Z_i} = \widetilde{W}_i[(\widetilde{W}^\top\widetilde{W})^{-1}\widetilde{W}^\top Z]
\end{eqnarray*}
where $\widetilde{W} = (W,X)$
\item \emph{Stage 2}: Substitute for $\hat{Z}_i$ for $Z_i$ in the original 
equation, estimate $\beta$ and $\gamma$ by ordinary least squares regression
of $Y$ on $X$ and $\hat{Z}$ as in the following equation. 
\begin{eqnarray*}
Y_i=X_i\beta + \widehat{Z_i}\gamma + \epsilon_i,  \quad {\rm for} 
\quad   i=1,\ldots,N
\end{eqnarray*}
\end{itemize}
\subsubsection{See Also}
For information about three stage least square regression, see 
\Sref{3sls} and \texttt{help(3sls)}.
For information about seemingly unrelated regression, see
\Sref{sur} and \texttt{help(sur)}.
\subsubsection{Quantities of Interest}
\subsubsection{Output Values}
The output of each Zelig command contains useful information which you may
view. For example, if you run:
\begin{verbatim}
z.out <- zelig(formula=fml, model = "2sls", data)
\end{verbatim}
\noindent then you may examine the available information in \texttt{z.out} by
using \texttt{names(z.out)}, see the draws from the posterior distribution of
the \texttt{coefficients} by using \texttt{z.out\$coefficients}, and view a default
summary of information through \texttt{summary(z.out)}. Other elements
available through the \texttt{\$} operator are listed below:
\begin{itemize}
\item \texttt{h}: matrix of all (diagonally stacked) instrumental variables.
\item \texttt{single.eq.sigma}: different $\sigma^2$s for each single equation?.
\end{itemize}

\begin{itemize}
\item \texttt{method}: Estimation method. 
\item \texttt{g}: number of equations.
\item \texttt{n}: total number of observations.
\item \texttt{k}: total number of coefficients.
\item \texttt{ki}: total number of linear independent coefficients.
\item \texttt{df}: degrees of freedom of the whole system.
\item \texttt{iter}: number of iteration steps.
\item \texttt{b}: vector of all estimated coefficients.
\item \texttt{t}: $t$ values for $b$.
\item \texttt{se}: estimated standard errors of $b$.
\item \texttt{bt}: coefficient vector transformed by $TX$.
\item \texttt{p}: $p$ values for $b$.
\item \texttt{bcov}: estimated covariance matrix of $b$.
\item \texttt{btcov}: covariance matrix of $bt$.
\item \texttt{rcov}: estimated residual covariance matrix.
\item \texttt{drcov}: determinant of \texttt{rcov}.
\item \texttt{rcor}: estimated residual correlation matrix.
\item \texttt{olsr2}: system OLS R-squared value.
\item \texttt{y}: vector of all (stacked) endogenous variables.
\item \texttt{x}: matrix of all (diagonally stacked) regressors.
\item \texttt{data}: data frame of the whole system (including instruments).
\item \texttt{TX}: matrix used to transform the regressor matrix.
\item \texttt{rcovformula}: formula to calculate the estimated residual covariance matrix.
\item \texttt{probdfsys}: system degrees of freedom to calculate probability values?.
\item \texttt{solvetol}: tolerance level when inverting a matrix or calculating a determinant.
\item \texttt{eq}: a list that contains the results that belong to the individual equations.
\item \texttt{eqnlabel*}: the equation label of the ith equation (from the labels list).
\item \texttt{formula*}: model formula of the ith equation.
\item \texttt{n*}: number of observations of the ith equation.
\item \texttt{k*}: number of coefficients/regressors in the ith equation (including the constant).
\item \texttt{ki*}: number of linear independent coefficients in the ith equation (including the
constant differs from k only if there are restrictions that are not cross equation).
\item \texttt{df*}: degrees of freedom of the ith equation.
\item \texttt{b*}: estimated coefficients of the ith equation. 
\item \texttt{se*}: estimated standard errors of $b$ of the ith equation.
\item \texttt{t*}: $t$ values for $b$ of the ith equation.
\item \texttt{p*}: $p$ values for $b$ of the ith equation.
\item \texttt{covb*}: estimated covariance matrix of $b$ of the ith equation.
\item \texttt{y*}: vector of endogenous variable (response values) of the ith equation.
\item \texttt{x*}: matrix of regressors (model matrix) of the ith equation.
\item \texttt{data*}: data frame (including instruments) of the ith equation.
\item \texttt{fitted*}: vector of fitted values of the ith equation.
\item \texttt{residuals*}: vector of residuals of the ith equaiton.
\item \texttt{ssr*}: sum of squared residuals of the ith equation.
\item \texttt{mse*}: estimated variance of the residuals (mean of squared errors) of the ith equation.
\item \texttt{s2*}: estimated variance of the residents($\hat{sigma}^2$) of the ith equation.
\item \texttt{rmse*}: estimated standard error of the reiduals (square root of mse) of the ith equation.
\item \texttt{s*}: estimated standard error of the residuals ($\hat{\sigma})$ of the ith equation.
\item \texttt{r2*}: R-squared (coefficient of determination).
\item \texttt{adjr2*}: adjusted R-squared value.



\end{itemize}


\begin{itemize}
\item \texttt{inst*}: instruments of the ith equation.
\item \texttt{h*}: matrix of instrumental variables of the ith equation. 
\end{itemize}
\subsubsection{Contributors}
The \texttt{2sls} function is adapted from The \texttt{systemfit} library by Jeff Hamann and Arne Henningsen. I fyou use this model, please cite:
\begin{verse}
\bibentry{HamaHenn05}
\end{verse}

\end{document}